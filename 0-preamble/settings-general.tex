%--------------------------------------------------------------
%               General settings for thesis   
%--------------------------------------------------------------

%--------------------GENERAL-SETTINGS--------------------------

\newlength{\abcd} % for ab..z string length calculation
% "\newlength" is a LaTeX macro which defines a new length register, which holds a length as number and can be used for calculations.

%\setlength{\parindent}{0cm}
% To remove the leading indent of the new paragraph.

% To automatically line break texttt.
\newcommand*\justify{%
  \fontdimen2\font=0.4em% interword space
  \fontdimen3\font=0.2em% interword stretch
  \fontdimen4\font=0.1em% interword shrink
  \fontdimen7\font=0.1em% extra space
  \hyphenchar\font=`\-% allowing hyphenation
}
\renewcommand{\texttt}[1]{%
  \begingroup
  \ttfamily
  \begingroup\lccode`~=`/\lowercase{\endgroup\def~}{/\discretionary{}{}{}}%
  \begingroup\lccode`~=`[\lowercase{\endgroup\def~}{[\discretionary{}{}{}}%
  \begingroup\lccode`~=`.\lowercase{\endgroup\def~}{.\discretionary{}{}{}}%
  \catcode`/=\active\catcode`[=\active\catcode`.=\active
  \justify\scantokens{#1\noexpand}%
  \endgroup
}

%---------------------------------------------------------------

%--------------------PAGE-LAYOUT-SETTINGS-----------------------

% Setting the page layout using the "geometry" package.
\geometry{
	a4paper,
	ignoremp,
	bindingoffset = 1cm, 
	textwidth     = 13.5cm,
	textheight    = 21.5cm,
	lmargin       = 3.5cm,
	tmargin       = 4cm   
}
% The page layout in the "geometry" package contains a total body (printable area) and margins: the total body consists of a body (text area) with an optional header, footer and marginal notes; the margins are left, right, top and bottom (for twosided documents, horizontal margins should be called inner and outer).

% The "a4paper" specifies the paper size by name.

% The "ignoremp" option disregards the marginal notes in determining the horizontal margins (defaults to true).

% The "bindingoffset" option removes a specified space from the lefthand-side of the page for oneside printing or the inner-side for twoside printing. This is useful if pages are bound by a press binding.

% The "textwidth" option the width of the body.

% The "textheight" option sets the height of the body (including footnotes and figures, excluding running head and foot).

% The "lmargin" option sets the left margin (for oneside printing) or inner margin (for twoside printing) of total body: the distance between the left or inner edge of the paper and that of total body.

% The "tmargin" option sets the top margin of the page.

%--------------------------------------------------------------

%--------------------SOURCES-FOR-COMMENTS----------------------

% The comments on LaTeX and its commands are based on the contents of https://latexref.xyz/, an unofficial reference manual for the LaTeX2e document preparation system.

% The comments on the classes, styles or packages (and their commands and options) come from the description provided on CTAN (https://www.ctan.org/) and from the official documentation of the different classes, styles or packages.

%--------------------------------------------------------------