%---------------------------------------------------------------
%				  Code related packages for thesis
%---------------------------------------------------------------

%------------------------USED-CODE-PACKAGES---------------------

%---------------------------------------------------------------

%----------------CODE-PACKAGES-LOADED-BY-CLASSICTHESIS----------

%\usepackage{listings} % WILL BE LOADED by "dia-classicthesis-ldpkg".
% The "listings" package is a source code printer for LaTeX. This package provides a more advanced code-formatting features as compared to the verbatim environment (the default to display code in LaTeX which generates an output in monospaced font).

%\usepackage{xcolor} % WILL BE LOADED by "classicthesis".
% The "xcolor" provides easy driver-independent access to several kinds of colors, tints, shades, tones, and mixes of arbitrary colors by means of color expressions like \color{red!50!green!20!blue}. It allows to select a document-wide target color model and offers tools for automatic color schemes, conversion between twelve color models, alternating table row colors, color blending and masking, color separation, and color wheel calculations.

%---------------------------------------------------------------

%---------------------UNUSED-CODE-PACKAGES----------------------

%\usepackage{verbatim}
% The "verbatim" package, alternative to the "listings" package, reimplements the LaTeX verbatim (the default to display code in LaTeX which generates an output in monospaced font) and verbatim* environments. This package provides also a comment environment that skips any commands or text between "\begin{comment}".

%----------------------------------------------------------------

%--------------------SOURCES-FOR-COMMENTS------------------------

% The comments on LaTeX and its commands are based on the contents of https://latexref.xyz/, an unofficial reference manual for the LaTeX2e document preparation system.

% The comments on the classes, styles or packages (and their commands and options) come from the description provided on CTAN (https://www.ctan.org/) and from the official documentation of the different classes, styles or packages.

%----------------------------------------------------------------