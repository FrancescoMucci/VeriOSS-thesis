%--------------------------------------------------------------
%                  Preamble for thesis   
%--------------------------------------------------------------

%--------------------DOCUMENT-CLASS----------------------------

% The document’s overall class is defined with the "\documentclass[<options list>]{<class-name>}" command, which is the first command in a LaTeX source file. The following document classes are built into LaTeX [ref: https://latexref.xyz/]: "article" for a journal article, a presentation, and miscellaneous general use; "book" for full-length books, including chapters and possibly including front matter, such as a preface, and back matter, such as an appendix; "letter" for mail; "report" for documents of length between an article and a book, such as technical reports or theses, which may contain several chapters; "slides" for slide presentations (rarely used nowadays, the "beamer" package is the most prevalent replacement). Many other document classes are available as separate packages.

\documentclass[twoside = false, openright, footinclude, headinclude, fleqn, 12pt, a4paper]{scrbook}
% The scrbook class provides provides features useful when preparing a book for publication. This class provides the "book"-like element of the "koma-script", a collection of replacements for the "article", "report" and "book" classes with emphasis on typography and versatility.

% DEACTIVATED
% Option "twoside": with this option the inner margin of one page is only half as wide as the corresponding outer margin; also the left and right margins are swapped on verso pages, so this option is ideal for double-sided printing. If the option is passed without a value, the value "true" is assumed, so two-sided printing is enabled. Deactivating the option with "twoside = false" leads to one-sided printing: the left and right margins are the same width.

% ACTIVATED
% Option "openright": a switch to indicate if chapters must start on a right-hand page. The default for the report class is no; for the book class it’s yes. Generate a page break and insert an interleaf page if needed to reach the next right-hand page in two-sided printing.

% ACTIVATED
% Options "headinclude" and "footinclude": these options cause the header or footer to be counted as part of the text.

% ACTIVATED
% Option "fleqn": equations are normally centred but, using the "fleqn" option, displayed equations are then left-justified.

% DEACTIVATED
% Option "titlepage": a switch to indicate if a titlepage has to be produced. For the article document class the default is not to make a separate titlepage. Using this option you can invoke \maketitle to creates titles on separate pages.

% DEACTIVATED
% Option "hidelinks": remove borders around clickable cross-references and hyperlinks.

%--------------------SUBDIRECTORIES-SETTING--------------------

% The following code declares a list of subdirectories so you can insert TeX files ("\input", "\include"), images ("\includegraphics"), or code ("\lstinputlisting") without specifying the path to the inserted file (if contained in one of these subdirectories).

\makeatletter
\providecommand*{\input@path}{}
\g@addto@macro\input@path{{0-preamble/}{0-preamble/basic/}{0-preamble/floats/}{0-preamble/math/}{0-preamble/code/}{0-preamble/commands/}{0-preamble/keywords/}{1-front/}{2-chapters/}{3-appendix/}{4-back/}{other/img/}{other/src/}}
\makeatother

% Our list of subdirectories is hold by "\input@path": an internal command; these kind of commands use "@" in their name, are mainly intended for package or class writers and normal users are prevented from accidentally redefining them. To be able to define or redifine such commands, you need to use  the commands "\makeatletter" and "\makeatother"; the first one ensures that the "@" character is seen as a letter when searching for command names; the last one disable the possibility to act on internal commands to avoid subsequent problems with command interpretation.

% The "\input@path" macro is normally undefined in LaTeX, but can be defined by some used package: for example, "graphics" and "graphicx" internally store the path specified in "\graphicspath" in "\Ginput@path" and locally sets "\input@path" to "\Ginput@path". We can therefore use "\providecommand" to define "\input@path" in the the case that it is undefined and then "\g@addto@macro" to extends its definition.

%--------------------------------------------------------------

%--------------------PACKAGES----------------------------------

%--------------------------------------------------------------
%                 Basic packages for thesis   
%--------------------------------------------------------------

%--------------------USED-BASIC-PACKAGES-----------------------

% To load a package the command "\usepackage[<options list>]{<pkg-name>}" must be used in the document preamble, between the "\documentclass" declaration and the "\begin{document}". To specify more than one package you can separate them with a comma, as in "\usepackage{<pkg1>,<pkg2>,...}", or use multiple "\usepackage" commands.

\usepackage{geometry}
% The "geometry" package provides a flexible and easy interface to page dimensions. You can change the page layout with intuitive parameters.

\usepackage[italian]{babel}
% The "babel" package which enables LaTeX to typeset in many different languages.

\usepackage[fixlanguage]{babelbib}
% The "babelbib" package enables the user to generate multilingual bibliographies in cooperation with babel: add the option "fixlanguage" to the package call so that all bibliographies are typeset in the document’s main language.

\usepackage[utf8]{inputenc}
% The package "inputenc" translates various standard and other input encodings into a LaTeX internal language. It is used by LaTeX to correctly interpret the characters entered in the editor. "utf8" is the input encoding which allows you to write the signs of numerous alphabets in the editor directly from the keyboard, avoiding having to load the encoding suitable for the language of the document each time.

\usepackage[square,numbers]{natbib} % LOAD BEFORE "hyperref".
% The "natbib" package permit to customise citations when using BibTeX (a tool for bibliography which comes bundled as standard with LaTeX). This package consist of a reimplementation of the LaTeX "\cite" command, to work with both author–year ("authoryear" option) and numerical citations ("numbers" option). The "square" option says that references are to be enclosed in square bracket rather than round parentheses. It is compatible with the packages: "babel", "index", "citeref", "showkeys", "chapterbib", "hyperref", "koma" and with the classes "amsbook" and "amsart".

\usepackage[format=hang, font=small, labelformat=simple, labelsep=colon]{caption}
% The "caption" package provides many ways to customise the captions in floating environments like figure and table, and cooperates with many other packages. With "format=hang" the caption text will be indented, so it will "hang" under the first line of the text. With "font=small" a small size font will be used for the caption. With the "labelformat=simple" the caption label will be typeset as a name and a number. With "labelsep=colon" the caption label and text will be separated by a colon and a space.

% COMMENTED OUT in "dia-classicthesis-ldpkg" because loaded with only the "footnote" option.
\usepackage[printonlyused, smaller]{acronym}
% The "acronym" provides an environment to define a list of acronym and to automatically build a table normally consisting of all defined acronyms, re-gardless if the the acronym was used in the text or not; to print only the acronym used at least once, use the option "printonlyused" (you can add to each acronym the the page number where it was first used by additionally specifying the option "withpage"). The first time you use a defined acronym, the full name of the acronym along with the acronym in brackets will be printed; if you specify the "footnote" option, the full name of the acronym will be printed as a footnote. The "smaller" option lets the acronyms appear a bit smaller (the "relsize" package is required). This package requires the "suffix" package, which in turn requires that it runs under e-TEX. The acronyms will be definied in the file "4-back/acronimi.tex" inside an "acronym" environment.

\usepackage{imakeidx} % LOAD BEFORE "hyperref".
% The "imakeidx" package enables the user to produce and typeset one or more indexes simultaneously with a document. This package rely upon "makeindex" TeX system programs to sort and format the index entries: to get an index you must first include "\makeindex" in the document preamble and compile the document. To add an entry to the index the command "\index{}" is used, where the word to be added is inserted as the parameter. Be careful, this won't print the word in the current position but only in the index. To obtain a linkable index, load "imakeidx" before "hyperref" (generally, "hyperref" should be loaded last, though there are some exceptions).

\usepackage{etoolbox}
% The "etoolbox" package is a toolbox of programming facilities geared primarily towards LaTeX class and package authors. It provides LaTeX frontends to some of the new primitives provided by e-TeX as well as some generic tools which are not related to e-TeX but match the profile of this package. For example, this package provides us with the "\AtEndPreamble{<code>}" which can be used to postpone the execution of the specified <code> at the end of the preamble.

\usepackage{hyphenat}
% The "hyphenat" package can disable all hyphenation or enable hyphenation of non-alphabetics or monospaced fonts; this package also enable hyphenation within words that contain non-alphabetic characters and hyphenation of text typeset in monospaced fonts. In particular TeX does not hyphenate already hyphenated word, such as "electromagnetic-endioscopy"; in this case, the "\hyp" command can be used to allow automatic hyphenation of compound words: e.g. "electromagnetic{\hyp}endioscopy".

\usepackage{needspace}
% The "needspace" package provides two commands, "\needspace" and "\Needspace", for reserving space to prevent a certain amount of material from being split over a page break. If there is not enough space, a "\newpage" is automatically inserted.

\usepackage{lipsum}
% The "lipsum" package gives you easy access to 150 paragraphs of the Lorem Ipsum dummy text.

%--------------------------------------------------------------

%--------------BASIC-PACKAGES-LOADED-BY-CLASSICTHESIS----------

%\usepackage[T1]{fontenc} % WILL BE LOADED by "dia-classicthesis-ldpkg".
% The "fontenc" package allows the user to select font encodings, and for each encoding provides an interface to font-encoding-specific commands for each font. Its most powerful effect is to enable hyphenation to operate on texts containing any character in the font. "T1" is the output encoding for writing in Italian and in many other western languages.

%\usepackage{textcomp} % WILL BE LOADED by "dia-classicthesis-ldpkg".
% The "textcomp" package supports the Text Companion fonts, which provide many text symbols (such as baht, bullet, copyright, musicalnote, onequarter, section, and yen), in the TS1 encoding. Note that the package has been adopted as part of the LATEX distribution.

%\usepackage{relsize} % WILL BE LOADED by "dia-classicthesis-ldpkg".
% The "relsize" package defines several commands to set font sizes relative to the current size: the basic command is "\relsize{<i>}" that change font size by <i> steps; starting from this, other command are defined such as "\larger[<i>]", "\smaller[<i>]" witch increase or reduce size by <i> steps (default 1).

%\usepackage{xspace} % WILL BE LOADED by "classicthesis".
% The "xspace" package provides the "\xspace" command that should create a space if the macro is used in text but no space if it is at the end of a sentence and followed by a full stop. This command should be used at the end of the definition of new commands which are designed to be used mainly in text. The technique used by this macro is not perfect, but works in a large proportion of cases.The package is part of the latex-tools bundle in the LaTeX required distribution.

%\usepackage{tocloft} % WILL BE LOADED by "classicthesis".
% The "tocloft" package provides means of controlling the typographic design of the "Table of Contents", "List of Figures" and "List of Tables"; also, new kinds of "List of..." can be defined.

%\usepackage{titlesec} % WILL BE LOADED by "classicthesis".
% The "titlesec" package provides an interface to sectioning commands for selection from various title styles; also includes a package to change the page styles when there are floats in a page and to assign headers/footers to individual floats.

%\usepackage{scrlayer-scrpage} % WILL BE LOADED by "classicthesis".
% The "scrlayer-scrpage" package is the part of the Koma-Script bundle that provides an end user interface to "scrlayer"; it allows the user to define and manage page styles by controlling page headers and footers.

%\usepackage{footmisc} % WILL BE LOADED by "classicthesis".
% The "footmisc" package permit to change the typesetting and the layout of footnotes. With the "para" option, footnotes will be to set inside paragraphs; with the "perpage" option footnotes will be numbered per page; with "multiple" option you can deal with multiple references to footnotes from the same place. This package also has a range of techniques for labelling footnotes with symbols rather than numbers.

%--------------------------------------------------------------

%--------------------UNUSED-BASIC-PACKAGES---------------------

%\usepackage{enumitem}
% This package provides user control over the layout of the three basic list environments: enumerate, itemize and description.

%\usepackage{comment}
% The "comment" package permit to selectively include/exclude pieces of text, allowing the user to define comments. All text between "\comment" ... "\endcomment" or "\begin{comment}" ... "\end{comment}" is treated as comments and discarded. The opening and closing commands should appear on a line of their own. No starting spaces, nothing after it.

%\usepackage{appendix}
% The "appendix" package provides some facilities for modifying the typesetting of appendix titles. The appendix package provides some commands that can be used in addition to the "\appendix" command.

%\usepackage{url}
% The "url" package defines "\url{...}" command intended for formatting email addresses, hypertext links, directories/paths, etc., which normally have no spaces. This command allows linebreaks at certain characters or combinations of characters and can usually be used in the argument to another command.  The font used may be selected using the \urlstyle command, and new url-like commands may be defined using \urldef. The command does not make hyper-links.

%---------------------------------------------------------------

%--------------------SOURCES-FOR-COMMENTS-----------------------

% The comments on LaTeX and its commands are based on the contents of https://latexref.xyz/, an unofficial reference manual for the LaTeX2e document preparation system.

% The comments on the classes, styles or packages (and their commands and options) come from the description provided on CTAN (https://www.ctan.org/) and from the official documentation of the different classes, styles or packages.

%--------------------------------------------------------------

%--------------------------------------------------------------
%    Floats (tables and figures) related packages for thesis   
%--------------------------------------------------------------

%--------------------FLOATS------------------------------------

% Floats are containers for things in a document that cannot be broken over a page. LaTeX by default recognizes "table" and "figure" floats, but you can define new ones of your own. Floats are there to deal with the problem of the object that won't fit on the present page and to help when you really don't want the object here just now. Floats are not part of the normal stream of text, but separate entities, positioned in a part of the page to themselves (top, middle, bottom, left, right, or wherever the designer specifies). They always have a caption describing them and they are always numbered so they can be referred to from elsewhere in the text. LaTeX automatically floats Tables and Figures, depending on how much space is left on the page at the point that they are processed. If there is not enough room on the current page, the float is moved to the top of the next page. This can be changed by moving the Table or Figure definition to an earlier or later point in the text, or by adjusting some of the parameters which control automatic floating.

% Source: https://en.wikibooks.org/wiki/LaTeX/Floats,_Figures_and_Captions.

%--------------------------------------------------------------

%--------------USED-FLOAT-PACKAGES-----------------------------

%--------------------------------------------------------------

%--------------FLOATS-PACKAGES-LOADED-BY-CLASSICTHESIS---------

%\usepackage{caption} % WILL BE LOADED by "dia-classicthesis-ldpkg".
% The "caption" package provides many ways to customise the captions in floating environments like figure and table, and cooperates with many other packages. Facilities include rotating captions, sideways captions, continued captions (for tables or figures that come in several parts).

%\usepackage{subfig} % WILL BE LOADED by "dia-classicthesis-ldpkg".
% The "subfig" package provides support for the inclusion of small, "sub", figures and tables. This package simplifies the positioning, captioning and labeling of such objects within a single figure or table environment and enable to continue a figure or table across multiple pages. The "subfig" package requires the "caption" and replaces the older "subfigure" package.

%--------------UNUSED-FLOAT-PACKAGES---------------------------

%\usepackage{float} % LOAD BEFORE "hyperref".
% The "float" package improves the interface for defining floating objects such as figures and tables. Introduces the boxed float, the ruled float and the plaintop float. You can define your own floats and improve the behaviour of the old ones. 

%\usepackage{wrapfig}
% The "wrapfig" package allows figures or tables to have text wrapped around them. Does not work in combination with list environments, but can be used in a parbox or minipage, and in twocolumn format. Supports the float package.

%\usepackage{lscape}
% The "lscape" package permit us to place selected parts of a document in landscape: modifies the margins and rotates the page contents but not the page number. Useful, for example, with large multipage tables, as it is compatible with longtable and supertabular.

%--------------------------------------------------------------

%--------------------TABLES-PACKAGES---------------------------

%--------------------------------------------------------------
%             Tables related packages for thesis   
%--------------------------------------------------------------

%--------------------USED-TABLES-PACKAGES----------------------

%--------------------------------------------------------------

%-------------TABLES-PACKAGES-LOADED-BY-CLASSICTHESIS----------

%\usepackage{tabularx} % WILL BE LOADED by "dia-classicthesis-ldpkg".
% The "tabularx" package defines the "tabularx" environment, an extension of "tabular" but, to set a table with the requested total width, modifies the widths of certain columns rather than the inter column space; the columns that may stretch are marked with the new token "X" in the preamble argument. This package requires the "array" package.

%\usepackage{booktabs} % WILL BE LOADED by "classicthesis".
% The "booktab" package enhances the quality of tables in LaTeX, providing extra commands as well as behind-the-scenes optimisation. From version 1.61, the package offers "longtable" compatibility.

%--------------------------------------------------------------

%--------------------UNUSED-TABLES-PACKAGES--------------------

%\usepackage{longtable} % LOAD BEFORE "hyperref".
% The "longtable" package allows you to write tables that continue to the next page. You can write captions within the table and headers and trailers for pages of table. Longtable arranges that the columns on successive pages have the same widths.

%\usepackage{multirow}
% The "multirow" package enable the construction of tables with cells that span more than one row of the table.

%---------------------------------------------------------------

%--------------------SOURCES-FOR-COMMENTS-----------------------

% The comments on LaTeX and its commands are based on the contents of https://latexref.xyz/, an unofficial reference manual for the LaTeX2e document preparation system.

% The comments on the classes, styles or packages (and their commands and options) come from the description provided on CTAN (https://www.ctan.org/) and from the official documentation of the different classes, styles or packages.

%---------------------------------------------------------------

%--------------------------------------------------------------

%--------------------FIGURES-PACKAGES--------------------------

%--------------------------------------------------------------
%             Figures related packages for thesis   
%--------------------------------------------------------------

%--------------------USED-FIGURES-PACKAGES---------------------

%--------------------------------------------------------------

%-------------FIGURES-PACKAGES-LOADED-BY-CLASSICTHESIS---------

%\usepackage[pdftex]{graphicx} % WILL BE LOADED by "dia-classicthesis-ldpkg".
% The "graphicx" package builds upon the "graphics" package and can be seen as an extended version of it: it provides a key-value interface for optional arguments to the command "\includegraphics". This packages rely on features that are not in TeX itself; these features must be supplied by the "driver" used to print the "dvi" file, but not all drivers support the same features: you can specify in the options which driver is being used (in our case "pdftex"), but normally this is not necessary; it will be enstablished automatically allowing the document to be portable between different systems.

%--------------------------------------------------------------

%--------------------UNUSED-FIGURES-PACKAGES-------------------

%--------------------------------------------------------------

%--------------------SOURCES-FOR-COMMENTS-----------------------

% The comments on LaTeX and its commands are based on the contents of https://latexref.xyz/, an unofficial reference manual for the LaTeX2e document preparation system.

% The comments on the classes, styles or packages (and their commands and options) come from the description provided on CTAN (https://www.ctan.org/) and from the official documentation of the different classes, styles or packages.

%--------------------------------------------------------------

%--------------------------------------------------------------

%--------------------SOURCES-FOR-COMMENTS-----------------------

% The comments on LaTeX and its commands are based on the contents of https://latexref.xyz/, an unofficial reference manual for the LaTeX2e document preparation system.

% The comments on the classes, styles or packages (and their commands and options) come from the description provided on CTAN (https://www.ctan.org/) and from the official documentation of the different classes, styles or packages.

%---------------------------------------------------------------

%---------------------------------------------------------------
%            Math related packages for thesis
%---------------------------------------------------------------

%--------------------USED-MATH-PACKAGES-------------------------

\usepackage[fleqn]{amsmath}
% The "amsmath" package is a LaTeX package that provides various extensions for improving the information structure and printing of documents containing mathematical formulas. The "fleqn" option position equations at a fixed indent from the left margin rather than centered in the text column.

\usepackage{amssymb}
% The "amssymb" package provides additional math symbols, like arrows, operators, special characters, geometric figures.

\usepackage{mathtools}
% The "mathtools" package is an extension package to amsmath. There are two things on "mathtools" agenda: (1) correct various bugs/deficiencies in amsmath until these are fixed by the AMS and (2) provide useful tools for mathematical typesetting for example the ability to write over arrows.

\usepackage{stmaryrd}
% The "stmaryrd" packages provides a number of new symbols for theoretical computer science, including ones for derivation of functional programming, process algebra, domain theory, linear logic, multisets and many more. It also fixes some features with AMS symbols and adds obvious variants of others.

\usepackage{amsthm}
% The "amsthm" package facilitates the kind of theorem setup typically needed in American Mathematical Society publications. This package provides an enhanced version of LaTeX’s "\newtheorem" command for defining theorem-like environments. The enhanced "\newtheorem" recognizes a "\theoremstyle" specification. The package also defines a proof environment that automatically adds a QED symbol at the end. If the "amsthm" package is used with a non-AMS document class and with the "amsmath" package, this must be loaded after "amsmath".

%---------------------------------------------------------------

%--------------------UNSED-MATH-PACKAGES------------------------

%\usepackage{mathpartir} 
% The "mathpartir" package provides macros for typesetting math formulas in mixed horizontal and vertical mode, e.g. fractions, inference rules and derivations.

%\usepackage{siunitx}
% The "siunitx" package handle all of the possible unit of measure related needs of LaTeX users.

%---------------------------------------------------------------

%--------------------SOURCES-FOR-COMMENTS-----------------------

% The comments on LaTeX and its commands are based on the contents of https://latexref.xyz/, an unofficial reference manual for the LaTeX2e document preparation system.

% The comments on the classes, styles or packages (and their commands and options) come from the description provided on CTAN (https://www.ctan.org/) and from the official documentation of the different classes, styles or packages.

%----------------------------------------------------------------

%---------------------------------------------------------------
%				  Code related packages for thesis
%---------------------------------------------------------------

%------------------------USED-CODE-PACKAGES---------------------

%---------------------------------------------------------------

%----------------CODE-PACKAGES-LOADED-BY-CLASSICTHESIS----------

%\usepackage{listings} % WILL BE LOADED by "dia-classicthesis-ldpkg".
% The "listings" package is a source code printer for LaTeX. This package provides a more advanced code-formatting features as compared to the verbatim environment (the default to display code in LaTeX which generates an output in monospaced font).

%\usepackage{xcolor} % WILL BE LOADED by "classicthesis".
% The "xcolor" provides easy driver-independent access to several kinds of colors, tints, shades, tones, and mixes of arbitrary colors by means of color expressions like \color{red!50!green!20!blue}. It allows to select a document-wide target color model and offers tools for automatic color schemes, conversion between twelve color models, alternating table row colors, color blending and masking, color separation, and color wheel calculations.

%---------------------------------------------------------------

%---------------------UNUSED-CODE-PACKAGES----------------------

%\usepackage{verbatim}
% The "verbatim" package, alternative to the "listings" package, reimplements the LaTeX verbatim (the default to display code in LaTeX which generates an output in monospaced font) and verbatim* environments. This package provides also a comment environment that skips any commands or text between "\begin{comment}".

%----------------------------------------------------------------

%--------------------SOURCES-FOR-COMMENTS------------------------

% The comments on LaTeX and its commands are based on the contents of https://latexref.xyz/, an unofficial reference manual for the LaTeX2e document preparation system.

% The comments on the classes, styles or packages (and their commands and options) come from the description provided on CTAN (https://www.ctan.org/) and from the official documentation of the different classes, styles or packages.

%----------------------------------------------------------------

%--------------------------------------------------------------
%                        Thesis style 
%--------------------------------------------------------------

%--------------------CLASSIC-THESIS-STYLE----------------------

% Classic Thesis Style is an easy-to-use template for Master’s or PhD thesis (Copyright (C) 2015 André Miede http://www.miede.de).

\usepackage{dia-classicthesis-ldpkg}
% Before loading the "classicthesis" style, you first need to load "dia-classicthesis-ldpkg", the package loader for this style. If you inspect the "dia-classicthesis-ldpkg.sty" file you will see a lot of "\RequirePackage" commands, this command, just like "\usepackage", load a package: the most significant difference between these two commands is that "\RequirePackage" can be used in a document before the "\documentclass" command, so this command is intended to be used in package and class files. 

% These are some of the packages preloaded by "dia-classicthesis-ldpkg" (for an exhaustive list please consult the file itself):
% - [T1]{fontenc} 		to get the output encoding for writing in Italian;        
% - {textcomp}			to get additional text symbols in the TS1 encoding;
% - {xspace}			to set the spacing after macros right;
% - {tabularx}			to get better tables;                                   
% - {caption} 			to take care of the caption fonts and sizes; 
% - {subfig} 			to enable subfigures in figures;
% - {listings} 			to obtain fine typesetting of code listings;
% - {hyperref}  		to produce hypertext links in the document;
% - {relsize} 			to scale font up or down;
% - {graphicx}			to get optional arguments for the "\includegraphics" command;
% - [footnote]{acronym} to get macros for handling all acronyms in the thesis (COMMENTED OUT!).

% ATTENTION: since "dia-classicthesis-ldpkg" loads the "hyperref" package, it is best to load this package loader after most other packages are already loaded.

% These are also some useful new commands defined in "dia-classicthesis-ldpkg" (for an exhaustive list please consult the file itself):
% - \ie 	for {i.\,e.}
% - \Ie 	for {I.\,e.}
% - \eg 	for {e.\,g.}
% - \Eg 	for {E.\,g.}
% - \etAl 	for {et al.\xspace}
% - \RA 	for {\ensuremath{\Rightarrow}}
% - \ra		for {\ensuremath{\rightarrow}}
% - \lra	for {\ensuremath{\leftrightarrow}}
% - \On		for {\ensuremath{O(n)}\xspace}
% - \Ologn	for {\ensuremath{O(\log n)}\xspace}
% - \Oone	for {\ensuremath{O(1)}\xspace}
  
\usepackage[eulerchapternumbers, linedheaders, subfig, beramono, eulermath, parts, dottedtoc, listings]{classicthesis}

% ACTIVATED
% Option "eulerchapternumbers": use figures from Hermann Zapf’s Euler math font for the chapter numbers. By default, old style figures from the Palatino font are used.

% ACTIVATED
% Option "linedheaders": changes the look of the chapter headings a bit by adding a horizontal line above the chapter title. The chapter number will also be moved to the top of the page, above the chapter title.

% ACTIVATED
% Option "subfig": if this option is specified, the setup for preloaded "subfig" package (which provides support for the inclusion of small, "sub", figures and tables) is enabled. This package simplifies the positioning, captioning and labeling of such objects within a single figure or table environment and to continue a figure or table across multiple pages. The "subfig" package requires the "caption" package by H.A. Sommerfeldt.

% ACTIVATED
% Option "beramono": loads Bera Mono as typewriter font. Default setting is using the standard CM typewriter font.

% ACTIVATED
% Option "eulermath": loads the awesome Euler fonts for math. Palatino is used as default font.

% ACTIVATED
% Option "parts": use this option if you use Part divisions in your document. This is necessary to get the spacing of the Table of Contents right.

% ACTIVATED
% Option "dottedtoc": sets pagenumbers flushed right in the table of contents.

% ACTIVATED
% Option "listings": if this option is specified, the setup for preloaded "listing" package is enabled. This package is a source code printer for LaTeX.

% DEFAULT
% Option "style": this offers a comfortable way of changing the look and feel easily. Default style is "classicthesis" . As a new feature, Lorenzo Pantieri’s "arsclassica" is available as well. As Lorenzo’s package is discontinued and with his permission, "classicthesis-arsclassica.sty" is now part of "classicthesis" and will be maintained here.

% DEACTIVATED
% Option "tocaligned": aligns the whole table of contents on the left side.

% DEACTIVATED
% Option "drafting" : prints the date and time at the bottom of each page, so you always know which version you are dealing with. Might come in handy not to give your Prof. that old draft.

% DEACTIVATED
% Option "manychapters": if you need more than nine chapters for your document, you might not be happy with the spacing between the chapter number and the chapter title in the Table of Contents. This option allows for additional space in this context. However, it does not look as “perfect” if you use "\parts" for structuring your document.

% DEACTIVATED (NOT WORKING: doesn't seem to work properly for tables and figures.)
% Option "floatperchapter": activates numbering per chapter for all floats such as figures, tables, and listings.

% The "classicthesis" style also load the following useful packages (for an exhaustive list please consult the file itself):
% - {xcolor} 			to access and to mix several kinds of colors;
% - {booktabs} 			to get commands to enhance the quality of tables;
% - {titlesec} 			to select various title styles for sectioning commands;
% - {scrlayer-scrpage}  to define and manage page styles by controlling page headers and footers;
% - {tocloft} 			to control the typography of the ToC and list of figures and tables;
% - {footmisc}			to change the typesetting of footnotes.

%---------------------------------------------------------------

%--------------------SOURCES-FOR-COMMENTS-----------------------

% The comments on LaTeX and its commands are based on the contents of https://latexref.xyz/, an unofficial reference manual for the LaTeX2e document preparation system.

% The comments on the classes, styles or packages (and their commands and options) come from the description provided on CTAN (https://www.ctan.org/) and from the official documentation of the different classes, styles or packages.

%---------------------------------------------------------------

%--------------------------------------------------------------
%         Basic packages for thesis to be loaded as last 
%--------------------------------------------------------------

%------------PACKAGES-TO-BE-LOADED-AFTER-CLASSICTHESIS---------

% Trying to load "tikz" (or "tcolorbox") before "dia-classicthesis-ldpkg" will cause an "option clash" error because both packages will load "graphicx" but with different options; loading "tikz" (or "tcolorbox") after "dia-classicthesis-ldpkg" solved the problem.

\usepackage{tikz} 
% The "tikz" package enables you to do nice figures in LaTeX: for example it can be used to draw automata.

\usepackage[most]{tcolorbox}
% The "tcolorbox" package provides an environment for coloured and framed text boxes with a heading line. Optionally, such a box may be split in an upper and a lower part; thus the package may be used for the setting of LaTeX examples where one part of the box displays the source code and the other part shows the output. Another common use case is the setting of theorems. The package supports saving and reuse of source code and text parts. The "most" option loads most of the "tcolorbox" libraries, including those required for creating breakable boxes, which are boxes that can automatically span across multiple pages.

%---------------------HYPERREF-PACKAGE-------------------------

% \usepackage{hyperref} % ALREADY LOADED by "dia-classicthesis-ldpkg".
% The "hyperref" package is used to handle cross-referencing commands in LaTeX to produce hypertext links in the document. Since this package redefine many LaTeX commands, as a rule of thumb, it is better to load this package as the last one (unless otherwise specified), to give it a fighting chance of not being over-written.

% Going into more detail, "hyperref" aims to cooperate with many other packages, but there are several possible sources for conflict such as: 
% - packages that manipulate the bibliographic mechanism (the recommended one is "natbib");
% - packages that changes "\label" and "\ref" macros;
% - packages that do anything serious with the analytical index; 
% - packages that do anything serious with sectioning commands and the toc. 
% To reduce the possibilies of conflict, some packages need to be loaded before "hyperref" and others after (for an exhaustive list see "hyperref" documentation).

% Some packages that need to be loaded BEFORE "hyperref": "float", "longtable", "ltabptch", "multind", "natbib", "prettyref", "setspace", "titleref".

% Some packages that need to be loaded AFTER "hyperref": "amsrefs", "arydshln", "dblaccnt", "ellipsis", "linguex", "cleveref".

%---------------------------------------------------------------

%--------------PACKAGES-TO-BE-LOADED-AFTER-HYPERREF-------------

\usepackage{ellipsis} % LOAD AFTER "hyperref".
% The "ellipsis" package fixes a problem in the way LaTeX handles ellipses ("\dots"); LaTeX always puts a tiny bit more space after "\dots" in text mode than before it, which results in the ellipsis being off-center when used between two words.

\usepackage[italian, noabbrev]{cleveref} % LOAD AFTER "hyperref" and AFTER "amsmath".
%The "cleveref" package enhances LaTeX’s label-referencing features, allowing the format of label-references to be determined automatically according to the “type” of cross-reference (equation, section, etc.) and the context in which the cross-reference is used: e.g. "\cref{eq1}" will be typeset as directly as "eq. 1" instead of simply "1" (obtained using "\ref{eq1}"). The "noabbrev" option disable the use of abbreviations in the default cross-reference names: e.g."\cref{eq1}" will be typeset as "equation (1)". The support for "babel" is available. Care must be taken when using "cleveref" in conjunction with other packages that modify LaTeX's referencing system: this package must be loaded last.

%---------------------------------------------------------------

%--------------------SOURCES-FOR-COMMENTS-----------------------

% The comments on LaTeX and its commands are based on the contents of https://latexref.xyz/, an unofficial reference manual for the LaTeX2e document preparation system.

% The comments on the classes, styles or packages (and their commands and options) come from the description provided on CTAN (https://www.ctan.org/) and from the official documentation of the different classes, styles or packages.

%---------------------------------------------------------------

%--------------------------------------------------------------

%--------------------SETTINGS----------------------------------

%--------------------------------------------------------------
%               General settings for thesis   
%--------------------------------------------------------------

%--------------------GENERAL-SETTINGS--------------------------

\newlength{\abcd} % for ab..z string length calculation
% "\newlength" is a LaTeX macro which defines a new length register, which holds a length as number and can be used for calculations.

%\setlength{\parindent}{0cm}
% To remove the leading indent of the new paragraph.

% To automatically line break texttt.
\newcommand*\justify{%
  \fontdimen2\font=0.4em% interword space
  \fontdimen3\font=0.2em% interword stretch
  \fontdimen4\font=0.1em% interword shrink
  \fontdimen7\font=0.1em% extra space
  \hyphenchar\font=`\-% allowing hyphenation
}
\renewcommand{\texttt}[1]{%
  \begingroup
  \ttfamily
  \begingroup\lccode`~=`/\lowercase{\endgroup\def~}{/\discretionary{}{}{}}%
  \begingroup\lccode`~=`[\lowercase{\endgroup\def~}{[\discretionary{}{}{}}%
  \begingroup\lccode`~=`.\lowercase{\endgroup\def~}{.\discretionary{}{}{}}%
  \catcode`/=\active\catcode`[=\active\catcode`.=\active
  \justify\scantokens{#1\noexpand}%
  \endgroup
}

%---------------------------------------------------------------

%--------------------PAGE-LAYOUT-SETTINGS-----------------------

% Setting the page layout using the "geometry" package.
\geometry{
	a4paper,
	ignoremp,
	bindingoffset = 1cm, 
	textwidth     = 13.5cm,
	textheight    = 21.5cm,
	lmargin       = 3.5cm,
	tmargin       = 4cm   
}
% The page layout in the "geometry" package contains a total body (printable area) and margins: the total body consists of a body (text area) with an optional header, footer and marginal notes; the margins are left, right, top and bottom (for twosided documents, horizontal margins should be called inner and outer).

% The "a4paper" specifies the paper size by name.

% The "ignoremp" option disregards the marginal notes in determining the horizontal margins (defaults to true).

% The "bindingoffset" option removes a specified space from the lefthand-side of the page for oneside printing or the inner-side for twoside printing. This is useful if pages are bound by a press binding.

% The "textwidth" option the width of the body.

% The "textheight" option sets the height of the body (including footnotes and figures, excluding running head and foot).

% The "lmargin" option sets the left margin (for oneside printing) or inner margin (for twoside printing) of total body: the distance between the left or inner edge of the paper and that of total body.

% The "tmargin" option sets the top margin of the page.

%--------------------------------------------------------------

%--------------------SOURCES-FOR-COMMENTS----------------------

% The comments on LaTeX and its commands are based on the contents of https://latexref.xyz/, an unofficial reference manual for the LaTeX2e document preparation system.

% The comments on the classes, styles or packages (and their commands and options) come from the description provided on CTAN (https://www.ctan.org/) and from the official documentation of the different classes, styles or packages.

%--------------------------------------------------------------

%---------------------------------------------------------------
%             Settings for thesis basic packages
%---------------------------------------------------------------

%--------------------CLASSICTHESIS-SETTINGS---------------------

% The "classicthesis" style adds a "\deactivateaddvspace" command at the start of list of figures, list of tables and list of listings files; this macro deactivate the grouping by chapter for the list entries. In order to not mess with the "classicthesis" code, we can define another macro that will cause "\deactiveateaddvspace" to do nothing.
\newcommand{\killdeactivateaddvspace}{\let\deactivateaddvspace\relax}
% Now we just need to add the former macro at the very start of the corresponding <list-of-x> file using the following command:
% - "\AtEndPreamble{\addtocontents{<list-of-x>}{\protect\killdeactivateaddvspace}}"

%---------------------------------------------------------------

%--------------------XSPACE-SETTINGS----------------------------

% The "\xspace" command normally produce a space before "]" and "}" but not before ")": using the "\xspaceaddexceptions" command we can set the behavior of "\xspace" so that it does not add space before any kind of brackets.
\xspaceaddexceptions{]\}>}

%---------------------------------------------------------------

%--------------------IMAKEIDX-SETTINGS--------------------------

\makeindex[columns=2, columnseprule, options={-s 0-preamble/basic/index-style.ist}]
% The command "\makeindex" is mandatory for the analytical index to work and can take some parameters to customize its appearance. By setting the "intoc" option an entry for this particular index is put in the table of contents. With the "columns" options you can set the number of columns in the index; if "columnseprule" it is set to true, a rule will appear between the columns. In order to customaze the index sorting and formatting you can specify a style file to use through the option "options={-s <style-file-name>.ist}".

\indexsetup{headers={\indexname}{\indexname}}
% The command "\indexsetup{<key-values list>}" allows further customisation of the analytical index. The default page header produced for the analytical index is not consistent with the header of the other chapters of the back matter (it's uppercase), so you can customize it using the option "headers={<left>}{<right>}" and exploiting the command '\indexname' which allows us to access the name associated with our analytical index.

\renewcommand{\dotfill}{\leavevmode\cleaders\hbox to 0.70em{\hss .\hss }\hfill\kern0pt }
% To adjust the spacing of dots produced by the command "\dotfill": we changes the default length of 0.33em to 0.70em to make the dots used in the analytical index consistent with those used in the table of contents and in the lists of figures, tables and listings.

%---------------------------------------------------------------

%--------------------CLEVEREF-SETTINGS--------------------------

\crefname{listing}{codice}{codici}
% When we refer to a labeled block of code, the "cleveref" package does not correctly translate "listing" into Italian, so we configure the translation using the "\crefname" command and providing both the singular and plural of the term to use.

%---------------------------------------------------------------

%--------------------TOCLOFT-SETTINGS---------------------------

\setlength{\cftbeforechapskip}{2pt} 
% Adds extra space between chapter entries in the table of contents.

%---------------------------------------------------------------

%--------------------TOCLOFT-SETTINGS---------------------------

% We define "mytextboxstyle", a style setting for `tcolorbox` that facilitates the creation of text boxes which can automatically continue across multiple pages.
\tcbset{
  mytextboxstyle/.style={
    breakable,              % allows the box to break across pages
    colback     = white,    % sets the background color to white
    left        = 2mm,      % sets the left padding
    right       = 2mm,      % sets the right padding
    top         = 2mm,      % sets the top padding
    bottom      = 1mm,      % sets the bottom padding
    arc         = 1mm,      % sets the corner arc radius
    boxrule     = 0.5mm,    % sets the frame rule thickness
    toptitle    = 0.5mm,    % sets the space above the title
    bottomtitle = 0mm,      % sets the space below the title
    titlerule   = 0mm,      % sets the thickness of the rule below the title
  }
}

% We define "chapterIntroBox", a text box for writing chapter introductions.
\newtcolorbox{chapterIntroBox}{
  mytextboxstyle, 							% inherit all settings from "mytextboxstyle"
  colframe 	= darkgray, 					% sets the frame color to dark gray
  title 	= Introduzione al capitolo, 	% sets the box title
}

% We define "chapterSummaryBox", a text box for writing chapter summaries.
\newtcolorbox{chapterSummaryBox}{
  mytextboxstyle, 									% inherit all settings from "mytextboxstyle"
  colframe 	= black,							    % sets the frame color to black
  title 	= Riassunto del capitolo e conclusioni, % sets the box title
}

% We define "chapterIntroBox", a text box for writing appendix introductions.
\newtcolorbox{appendixIntroBox}{
  mytextboxstyle, 						  	% inherit all settings from "mytextboxstyle"
  colframe 	= gray,							% sets the frame color to gray
  title 	= Introduzione all'appendice,	% sets the box title
}

% We define "myTextBox", a simple box for writing text with a parametric title and customizable options.
\newtcolorbox{myTextBox}[2][]{
  mytextboxstyle, 	% inherit all settings from "mytextboxstyle"
  title = #2,     	% title of the box, passed as mandatory the second argument
  #1  				% first optional argument for additional settings
}

%--------------------ACRONYM-SETTINGS---------------------------

% Using tha package "acronym", any acronym printed by "\acs" is formatted by "\acsfont"; any acronym printed by "\acf" is formatted by "\acffont" and the included acronym is formatted by "\acfsfont" (and "\acsfont)".

% Italics font will be used when we prints the full name and the acronym in brackets.
%\renewcommand*{\acffont}[1]{\emph{#1}}

% Smaller bold font will be used for all acronyms (we can remove the package option "smaller").
%\renewcommand*{\acsfont}[1]{\textsmaller{\textbf{#1}}}

%---------------------------------------------------------------

%--------------------ENUMITEM-SETTINGS--------------------------

%\setlist{topsep=1pt, parsep=0.5pt, itemsep=0.5pt}
% To obtain more compressed list using the "enumitem" package.

%---------------------------------------------------------------

%--------------------SOURCES-FOR-COMMENTS-----------------------

% The comments on LaTeX and its commands are based on the contents of https://latexref.xyz/, an unofficial reference manual for the LaTeX2e document preparation system.

% The comments on the classes, styles or packages (and their commands and options) come from the description provided on CTAN (https://www.ctan.org/) and from the official documentation of the different classes, styles or packages.

%---------------------------------------------------------------

%---------------------------------------------------------------
%       Settings for floats related packages for thesis
%---------------------------------------------------------------

%--------------------TABLES-RELATED-PACKAGES-SETTINGS-----------

%---------------------------------------------------------------
%       Settings for tables related packages for thesis
%---------------------------------------------------------------

%--------------------TABLES-GENERAL-SETTINGS--------------------

\setlength{\extrarowheight}{3pt}
% To increase table row height.

\numberwithin{table}{chapter}
% To number tables by chapter.

% To group the entries in the list of tables by chapter.
\AtEndPreamble{\addtocontents{lot}{\protect\killdeactivateaddvspace}}

%---------------------------------------------------------------

%--------------------SOURCES-FOR-COMMENTS-----------------------

% The comments on LaTeX and its commands are based on the contents of https://latexref.xyz/, an unofficial reference manual for the LaTeX2e document preparation system.

% The comments on the classes, styles or packages (and their commands and options) come from the description provided on CTAN (https://www.ctan.org/) and from the official documentation of the different classes, styles or packages.

%---------------------------------------------------------------

%---------------------------------------------------------------

%--------------------FIGURES-RELATED-PACKAGES-SETTINGS----------

%---------------------------------------------------------------
%       Settings for figures related packages for thesis
%---------------------------------------------------------------

%--------------------FIGURES-GENERAL-SETTINGS-------------------

\numberwithin{figure}{chapter}
% To number figueres by chapter.

% To group the entries in the list of figures by chapter.
\AtEndPreamble{\addtocontents{lof}{\protect\killdeactivateaddvspace}}

%\graphicspath{{other/img/}} % UNNECESSARY because in "preamble-main" the internal command "\input@path" was set to holds our list of subdirectories.
% The command "\graphicspath{list of directories inside curly braces}" declare a list of directories to search for graphics files. This allows you to later say something like "\includegraphics{image-name}" instead of having to give its path.

%---------------------------------------------------------------

%--------------------SOURCES-FOR-COMMENTS-----------------------

% The comments on LaTeX and its commands are based on the contents of https://latexref.xyz/, an unofficial reference manual for the LaTeX2e document preparation system.

% The comments on the classes, styles or packages (and their commands and options) come from the description provided on CTAN (https://www.ctan.org/) and from the official documentation of the different classes, styles or packages.

%----------------------------------------------------------------

%---------------------------------------------------------------

%--------------------SOURCES-FOR-COMMENTS-----------------------

% The comments on LaTeX and its commands are based on the contents of https://latexref.xyz/, an unofficial reference manual for the LaTeX2e document preparation system.

% The comments on the classes, styles or packages (and their commands and options) come from the description provided on CTAN (https://www.ctan.org/) and from the official documentation of the different classes, styles or packages.

%----------------------------------------------------------------

%---------------------------------------------------------------
%       Settings for code related packages for thesis
%---------------------------------------------------------------

%--------------------LISTING-DOCS---------------------------------

% The package "listing" supports the insertion of code snippets, code segments and listings of stand alone files: snippets are placed inside paragraphs and the others as separate paragraphs.

% The command to create code snippets is "\lstinline[language=<name>]!<insert-code-here>!"; the exclamation marks delimit the code and can be replaced by any character not in the code. 

% The following environment can be used to create source code segments: \begin{lstlisting}[caption={<insert-caption>}, label={<insert-label>}, language=<name>].

% The following command an be used to pretty-print the lines from <x> to <x+y> of the specified files: \lstinputlisting[caption={<insert-caption>}, label={<insert-label>}, language=<name>, firstline=<x>, lastline=<x+y>]{<relative-path>/<file-name>.<ext>}.

%-----------------------------------------------------------------

%--------------------XCOLOR-SETTINGS------------------------------

% The command "\definecolor" is used to define new colours in rgb format that will later be used for code colouring.

% Background color for code segments and listings of stand alone files.
\definecolor{verylightgray}{rgb}{0.97,0.97,0.97}

%-----------------------------------------------------------------

%--------------------LISTING-SETTINGS-----------------------------

% Setting the caption label for listings (default is "Listing").
\renewcommand{\lstlistingname}{Codice}

% Setting the header name for the list of listings (default is "Listings").
\renewcommand{\lstlistlistingname}{Elenco dei codici}

% To group the entries in the list of listings by chapter.
\AtEndPreamble{\addtocontents{lol}{\protect\killdeactivateaddvspace}}

% UNNECESSARY since we specified the "listings" option for "classicthesis".
% To include the word "Codice" for each entry in the list of listings.
%\makeatletter
%\def\l@lstlisting#1#2{\@dottedtocline{1}{0em}{4em}{Codice #1}{#2}}
%\makeatother
% To be precise, we redifined the internal macro "\l@lstlisting" to change the format of each entry using the command "\@dottedtocline" which is normally used internally by LaTeX to format a line in the table of contents and list of figures/tables. 
% When we define a new listing, these arguments will be passed to "\l@lstlisting":
% - #1: the pair <number> <caption> of the listing;
% - #2: <page-number> of the listing.
% The command "\@dottedtocline" takes the following arguments:
% - <section-level-num>: the level of the entry (1 for first level, 2 for sub-entry);
% - <indent>: the entry indentation (0 for no indentation);
% - <numwidth>: the space between <number> and <caption> of the entry (4em is about 1.6cm);
% - <text>: the text of the entry (Codice <number> <caption>);
% - <pagenumber>: the page number associated with the entry.

%-----------------------------------------------------------------

%--------------------LISTING-STYLE-DEFINITION---------------------

% The parameter "language=<name>" enables code highlighting for the particular programming: if the specified language is supported, keywords are in boldface font and comments are italicized. If you don’t like these settings, the "listing" package enable customization of code formatting via style definible using the command "\lstdefinestyle{<style-name>}{<key=value list>}".

\lstdefinestyle{mycodestyle}{
% Code settings
	basicstyle      = \footnotesize\ttfamily, % Size and font used for the code.
	breaklines      = true, % Sets automatic line breaking of long lines.
	tabsize         = 2, % Sets tabulation as 2 spaces.
	keepspaces      = true, % Keeps spaces in code, useful for keeping indentation of code.
% Frame settings
	frame           = single, % Sets a frame around the code.
	framerule       = 0pt, % Frame with no rules.
	backgroundcolor = \color{verylightgray}, % Frame background color.
	frameround      = ffff, % Straight corner for frame.
% Line numbers settings
	numbers         = left, % Print line numbers on the left.
	numbersep       = 9pt, % Line numbers are 9pt from the code.
	stepnumber      = 1, % The step between two line numbers it's 1: each line is numbered).
	numberstyle     = \scriptsize\color{gray}, % Size and color of the line numbers.
% Caption settings
    captionpos      = b, % Sets the caption position to bottom.
    numberbychapter = true, % Sets how listing will be numbered (by chapter or continously).
% Positioning settings
	aboveskip       = \bigskipamount, % Sets the vertical space above displayed listings.
	belowskip       = \smallskipamount % Sets the vertical space below displayed listings.
}

\lstset{style=mycodestyle}
% To enables the style "mycodestyle".

%-----------------------------------------------------------------

%--------------------LANGUAGES-HIGHLIGHTING-----------------------

\input{code-highlighting/solidity}

%---------------------------------------------------------------
%              My personal Java highlightining
%---------------------------------------------------------------

%\usepackage{listings, xcolor}

\lstdefinelanguage{MyJava}{
language = Java,
% Keywords:
keywordstyle = \color{Brown},
morekeywords = {@interface},
% Strings:
stringstyle = \color{Emerald},
% Comments:
commentstyle = \color{gray}, 
% Abstract types:
morekeywords = [2]{List, BookRepository, BookService, BookWebController, IsbnConstraints, TitleConstraints, AuthorsConstraints, Default, ConstraintViolation, Set, BindingResult, Model, WebSecurityConfigurerAdapter, PasswordEncoder, BookDataMapper, APageObject, WebDriver, WebElement, MongoClient}, 
keywordstyle = [2]\color{Cyan},
% Concrete types:
morekeywords = [3]{String, Book, BookData, IsbnData, Optional, Sort, MyBookService, MyBookWebController, MyWebSecurityConfiguration, BCryptPasswordEncoder, AuthenticationManagerBuilder, Exception, HttpSecurity, WebSecurity, MockMvc, BookHomeViewTest, WebClient, SilentCssErrorHandler, HtmlPage, HtmlHeader, BookViewTestingHelperMethods, MyBookWebControllerExceptionHandler, MyWebSecurityConfigurationTest, HtmlForm, Date, Calendar, HtmlUnitDriver, SilentHtmlUnitDriver, PageFactory, MyPage, BookHomePage, BookListPage, BookNewPage, BookSearchByIsbnPage, BookSearchByTitlePage, BookEditPage, UnknownErrorPage, MyErrorPage, BookAlreadyExistErrorPage, InvalidIsbnErrorPage, BookNotFoundErrorPage, BookEditViewIT, BookListViewIT, MongoClients, SpringBookshelfApplicationE2E, ChromeOptions, ChromeDriver, Integer, System, Document, SpringRunner},
keywordstyle = [3]\color{blue},
% Annotations:
morekeywords = [4]{@Repository, @Service, @Over, @SuppressWarnings, @Override, @NotNull, @NotBlank, @Pattern, @ISBN, @Test, @GetMapping, @PostMapping, @Validated, @Valid, @Value, @Autowired, @Bean, @Configuration, @EnableWebSecurity, @Retention, @WithMockUser, @WithMockAdmin, WithMockAdmin, RetentionPolicy, @Controller, @ControllerAdvice, @ExceptionHandler, @ResponseStatus, @WebMvcTest, @MockBean, @Before, @FindBy, @After, @TestPropertySource, @RunWith},
keywordstyle = [4]\color{magenta},
% Exceptions:
morekeywords = [5]{BookNotFoundException, BookAlreadyExistException, InvalidIsbnException},
keywordstyle = [5]\color{red},
% Assertions:
morekeywords = [6]{when, thenReturn, assertThat, hasSize, isEqualTo, hasToString, isCloseTo, isEmpty, contains},
keywordstyle = [6]\color{green},
% Enums:
morekeywords = [7]{HttpStatus, ChronoUnit},
keywordstyle = [7]\color{Orchid},
}

%---------------------------------------------------------------
%               My personal HTML highlightining
%---------------------------------------------------------------

%\usepackage{listings, xcolor}

\lstdefinelanguage{MyHTML}{
language = HTML,
% Keywords:
alsoletter = {-, :}, % Special characters as keywords.
keywordstyle = \color{RoyalBlue},
morekeywords = {section, header, nav, footer, th:block}, % Keywords from Bootstrap
% Arguments inside delimeters:
deletekeywords = {class, id, text, type, name, method, lang}, % Ex default keywors
morekeywords = [2]{class, id, xmlns:th, th:text, th:fragment, th:replace, th:include, th:insert, th:classappend, th:block, th:if, xmlns:sec, sec:authorize, aria-hidden, th:action, th:object, type, name, method, lang, required, th:id, th:name, th:field, th:placeholder, th:errors, data-dismiss, aria-label, th:each, data-toggle, th:data-target},
keywordstyle = [2]\color{Sepia},
% Strings:
stringstyle = \color{Emerald},
% Comments:
commentstyle = \color{lightgray},
morecomment = [s]{<!-}{-->},
% Tags:
tagstyle=\color{gray},
}


%-----------------------------------------------------------------

%--------------------SOURCES-FOR-COMMENTS-------------------------

% The comments on LaTeX and its commands are based on the contents of https://latexref.xyz/, an unofficial reference manual for the LaTeX2e document preparation system.

% The comments on the classes, styles or packages (and their commands and options) come from the description provided on CTAN (https://www.ctan.org/) and from the official documentation of the different classes, styles or packages.

%-----------------------------------------------------------------

%---------------------------------------------------------------
%          Settings for math related packages for thesis
%---------------------------------------------------------------

%------------------------AMSTHM-DOCS----------------------------

% The "\newtheorem{<env-name>}{<head-text>}" command requires as the first argument the environment name and, as the second one, the heading text: e.g. "\newtheorem{thm}{Theorem}" means that in the document you can use "\begin{thm} ... \end{thm}" to produce a theorem environment with "Theorem" as heading text. 

% By default an automatically generated number will be assigned to the theorem; as optional argument we can use "[chapter]" or "[section]", so that the theorem will be numbered by chapter or by section: the counter will be reset to 0 whenever the parent counter chapeter/section is incremented and the theorem heading will have the chapter/section number prepended. You can also number, for example, lemmas and corollaries by theorem using the corresponding environment name (e.g. [thm]) as optional parameter.

% Each kind of theorem-like environment (e.g. theorem, lemma) is numbered independently: if you have one lemmas and then two theorems, they will be numbered as follow: "Lemma 1", "Lemma 2", "Theorem 1". If you want lemmas and theorems to share the same numbering sequence (e.g "Lemma 1", "Lemma 2", "Theorem 3"), then you should add the <env-name> used for the Theorem environment as secondo parameter for the Lemma environment definition: "\newtheorem{lem}[thm]{Lemma}"

% The theorem styles provide different degrees of visual emphasis corresponding to their relative importance. The "plain" style is the default, it use italic text and has extra space above and below. The "definition" use upright text and has extra space above and below. The "remark" use upright text and have no extra space above or below. To specify different styles, divide your "\newtheorem" commands into groups and preface each group with the appropriate "\theoremstyle".

% If "\newtheorem*" is used instead of "\newtheorem", numbers will not be generated automatically for any of the theorems in the document.

% By placing a "\swapnumbers" command at the beginning of the list of "\newtheorem" statements that should be affected, the theorem number will be at the beginning of the heading instead of at the end, for example “1.4 Theorem” instead of “Theorem 1.4”.

% The predefined proof environment produces the heading “Proof” with appropriate spacing and punctuation. An optional argument of the proof environment allows you to substitute a different name for the standard “Proof”: e.g. if you want the proof heading to be, say, "Proof of the Main Theorem", then write "\begin{proof}[Proof of the Main Theorem]". A QED symbol is automatically appended at the end of a proof environment. Placement of the QED symbol can be problematic if the last part of a proof environment is a displayed equation or list environment: in that case put a "\qedhere" command at the place where the QED symbol should appear.

%---------------------------------------------------------------

%------------------AMSTHM-PLAIN-ENVIRONMENTS--------------------

\theoremstyle{plain}

% Definition of "\newplaintheorem{<env-name>}{<head-text>}" for defining a "plain" theorem environment numbered according to a passed parameter.
\newtheorem{innerplaintheorem}{\plaintheoremname}
\providecommand{\plaintheoremname}{}
\newcommand{\newplaintheorem}[2]{
  \newenvironment{#1}[1]{
   \renewcommand\plaintheoremname{#2}
   \renewcommand\theinnerplaintheorem{##1}
   \innerplaintheorem}
   {\endinnerplaintheorem}
}

% Proposition (or statement): is a sentence that is either true or false but not both.
\newtheorem{proposizione}{Proposizione}[chapter]

% Proposition numbered according to a passed parameter.
\newplaintheorem{proposizione-num}{Proposizione}
% E.g. "\being{proposizione-num}{8.2.3} ... \end{proposizione-num}".

% Lemma (or helping-theorem or auxiliary-theorem): a true proposition used as a stepping stone to prove other proposition.
\newtheorem{lemma}{Lemma}[chapter]

% Lemma environment numbered according to a passed parameter.
\newplaintheorem{lemma-num}{Lemma}

% Theorem: a proposition that has been proven to be true.
\newtheorem{teorema}{Teorema}[chapter]

% Theorem environment numbered according to a passed parameter.
\newplaintheorem{teorema-num}{Teorema}

% Corollary: a true proprosition that is a simple deduction from a theorem or proposition.
\newtheorem{corollario}{Corollario}[chapter]

% Corollary environment numbered according to a passed parameter.
\newplaintheorem{corollario-num}{Corollario}

% Conjecture: a proposition believed to be true, but for which we have no proof.
\newtheorem{congettura}{Congettura}[chapter]

% Conjecture environment numbered according to a passed parameter.
\newplaintheorem{congettura-num}{Congettura}

%---------------------------------------------------------------

%----------------AMSTHM-DEFINITION-ENVIRONMENTS-----------------
  
\theoremstyle{definition}

% Definition of "\newdefinitiontheorem{<env-name>}{<head-text>}" for defining a "definition" theorem environment numbered according to a passed parameter.
\newtheorem{innerdefinitiontheorem}{\definitiontheoremname}
\providecommand{\definitiontheoremname}{}
\newcommand{\newdefinitiontheorem}[2]{
  \newenvironment{#1}[1]{
   \renewcommand\definitiontheoremname{#2}
   \renewcommand\theinnerdefinitiontheorem{##1}
   \innerdefinitiontheorem}
   {\endinnerdefinitiontheorem}
}

% Definition: an explanation of the mathematical meaning of a word.
\newtheorem{definizione}{Definizione}[chapter]

% Definition environment numbered according to a passed parameter.
\newdefinitiontheorem{definizione-num}{Definizione}

% Axiom: a basic assumption about a mathematical situation (a statement we assume to be true).
\newtheorem{assioma}{Assioma}[chapter]

% Axiom environment numbered according to a passed parameter.
\newdefinitiontheorem{assioma-num}{Assioma}
  
% Problem
\newtheorem{problema}{Problema}[chapter]

% Problem environment numbered according to a passed parameter.
\newdefinitiontheorem{problema-num}{Problema}

% Example
\newtheorem{esempio}{Esempio}[chapter]

% Example environment numbered according to a passed parameter.
\newdefinitiontheorem{esempio-num}{Esempio}

% Exercise
\newtheorem{esercizio}{Esercizio}[chapter]

% Exercise environment numbered according to a passed parameter.
\newdefinitiontheorem{esercizio-num}{Esercizio}

% Algorithm
\newtheorem{algoritmo}{Algoritmo}[chapter]

% Algorithm environment numbered according to a passed parameter.
\newdefinitiontheorem{algoritmo-num}{Algoritmo}

%---------------------------------------------------------------

%------------------AMSTHM-REMARK-ENVIRONMENTS-------------------

\theoremstyle{remark}

% Definition of "\newremarktheorem{<env-name>}{<head-text>}" for defining a "remark" theorem environment numbered according to a passed parameter.
\newtheorem{innerremarktheorem}{\remarktheoremname}
\providecommand{\remarktheoremname}{}
\newcommand{\newremarktheorem}[2]{
  \newenvironment{#1}[1]{
   \renewcommand\remarktheoremname{#2}
   \renewcommand\theinnerremarktheorem{##1}
   \innerremarktheorem}
   {\endinnerremarktheorem}
}

% Note
\newtheorem{nota}{Nota}[chapter]

% Note environment numbered according to a passed parameter.
\newremarktheorem{nota-num}{Nota}

\newtheorem{futuro}{Sviluppo Futuro}[chapter]
\newtheorem{limitazione}{Limitazione}[chapter]
\newtheorem{p2k}{Protocollo P2K}[chapter]
\newtheorem{domanda}{Domanda}[chapter]
\newtheorem{assunzioni}{Assunzioni}[chapter]
\newtheorem{requisiti}{Requisiti}[chapter]
\newtheorem{security}{Nota di Security}[chapter]

%---------------------------------------------------------------

%--------------------TIKZ-AUTOMATA-SETTINGS---------------------

% Settings necessary to draw finite state automata using the "tikz" package.

\usetikzlibrary{arrows, arrows.meta, automata, positioning, shapes} 
% Import useful "tikz" libraries.

\tikzset{elliptic state/.style={draw, ellipse, font=\footnotesize}}
% To draw elliptic states using the "footnotesize" font for state names.

\tikzset{every edge/.append style={font=\footnotesize}}
% To use the "footnotesize" font also for edge labels.

%---------------------------------------------------------------

%--------------------SOURCES-FOR-COMMENTS-----------------------

% The comments on LaTeX and its commands are based on the contents of https://latexref.xyz/, an unofficial reference manual for the LaTeX2e document preparation system.

% The comments on the classes, styles or packages (and their commands and options) come from the description provided on CTAN (https://www.ctan.org/) and from the official documentation of the different classes, styles or packages.

%----------------------------------------------------------------

%--------------------------------------------------------------

%--------------------COMMANDS----------------------------------

%--------------------------------------------------------------
%                Basic new commands for thesis
%--------------------------------------------------------------

%----------------------SMART-CLEAR-PAGE------------------------

% We defines the new command "\smartClearPage", it will execute "\cleardoublepage" if the twoside option is active (common in book class for double-sided printing), and "\clearpage" otherwise (typical for single-sided documents like articles or reports).

\makeatletter
\newcommand{\smartClearPage}{%
	\if@twoside%
		\cleardoublepage%
	\else%
		\clearpage%
	\fi%
}
\makeatother

% The "\clearpage" command can be used to end the current page and start a new one. This command is typically used to ensure that a new chapter or section starts on a fresh page, or to separate distinct parts of your document. This command doesn't care about whether the page is odd or even, so the new page will immediately follow the current one.

% The "\cleardoublepage" command is similar to "\clearpage", but it goes a step further for documents that are double-sided (like books or theses). When you use "\cleardoublepage", LaTeX will end the current page and then insert blank pages as necessary to ensure that the next page is a right-hand page. This is important in double-sided documents to maintain consistency, where new chapters traditionally start on the right-hand side.

% The "\makeatletter" and "\makeatother" commands are used to change the category code of "@" so that it can be used in command names. This is necessary because "\if@twoside" contains an "@" symbol, which is not normally allowed in command names in LaTeX's user mode.

%--------------------------------------------------------------

%----------------------ADD-TOC-LINE----------------------------

% We define the "\addTocLine{<unit>}{<text>}" command for adding entries to the table of contents. The firts argument is expected to be the name of a sectional unit: "part", "chapter", "section", "subsection", etc. The second argument is the text of the entry.

\newcommand{\addTocLine}[2]{%
	\phantomsection%
	\addcontentsline{toc}{#1}{#2}%
}

% The "\phantomsection" command, from the "hyperref" package, sets an anchor at its location; without it, the hyperlink in the table of contents might lead to an incorrect or unintended location.

% The "\addcontentsline" command is used to create the corresponding line inside the toc.

%--------------------------------------------------------------

%----------------------ADD-TOC-CHAPTER-------------------------

% We define the "\addTocChapter{<text>}" command for adding unnumbered chapter entries to the table of contents. The only argument is the text of the entry.

\newcommand{\addTocChapter}[1]{%
    \smartClearPage%
	\addTocLine{chapter}{#1}%
}

% The use of "\clearpage" (or "\cleardoublepage" for double-sided documents) is essential before adding an unnumbered chapter to the ToC: without it, the hyperlink created by the "hyperref" package might incorrectly point to the previous page, rather than to the page where the new chapter actually begins.

%--------------------------------------------------------------

%----------------------ADD-PDF-BOOKMARK------------------------

% We define the "\addPdfBookmark[<level>]{<text>}{<name>}" command for creates a bookmark with the specified <text> and at the given <level> (default is 0). The third argument is name of the internal anchor (so it must be unique).

\newcommand{\addPdfBookmark}[3][0]{%
	\phantomsection%
	\pdfbookmark[#1]{#2}{#3}%
}

% The "\phantomsection" command, from the "hyperref" package, sets an anchor at its location; without it, the hyperlink in the table of contents might lead to an incorrect or unintended location.

% The "\pdfbookmark" command, from the "hyperref" package, creates the bookmark.

%--------------------------------------------------------------

%----------------------ADD-CHAPTER-PDF-BOOKMARK----------------

% We define the "\addChapterBookmark{<text>}{<name>}" command for creates a bookmark for a chapter (at level 0) with the specified <text>. The third argument is name of the internal anchor (so it must be unique).

\newcommand{\addChapterBookmark}[2]{%
    \smartClearPage%
    \addPdfBookmark{#1}{#2}%
}

% The use of "\clearpage" (or "\cleardoublepage" for double-sided documents) is essential before adding a bookmark for an unnumbered chapter: without it the hyperlink created by the "hyperref" package might incorrectly point to the previous page, rather than to the page where the new chapter actually begins.

%--------------------------------------------------------------

%----------------------SMART-SECTION-CLEAR---------------------

% The "\smartSectionClear" command is designed to be used before starting a new section. This ensures that if the section is likely to start close to the bottom of the current page, the command will force a page break, causing the section to begin at the top of the next page. This helps in avoiding situations where a new section or a paragraph starts at the very bottom of a page, leaving little room for text and making the document look cluttered.

% First version:
%\newcommand{\smartSectionClear}{%
%	\ifdim\dimexpr\pagegoal-\pagetotal\relax<3\baselineskip%
%		\clearpage%
%	\fi
%}

% The first version of the "smartSectionClear" command estimates the space left on the page and inserts a "\clearpage" if there is less than 3 lines of space remaining. Using "\clearpage" is essential before a section that starts on a new page to ensure correct hyperlinking. Without it, the hyperlink created by the "hyperref" package might incorrectly point to the previous page, rather than the page where the new section actually begins.

% PROBLEM with the first version: on rare occasions, it adds a "\clearpage" even if there is enough space and I do not understand why.

% Second version:
%\newcommand{\smartSectionClear}{%
%	\needspace{3\baselineskip}
%}

% The second version of the "smartSectionClear" command use the "\needspace" command, from the "needspace" package, to check if there is enough space left on the current page for approximately the height of three lines of text. If there isn't enough space for these three lines, a "\newpage" is automatically inserted. The \newpage command simply starts a new page, while the "\clearpage" command, used in the previous version, not only starts a new page but also ensures that all pending floats are placed before the new page begins.

% PROBLEM with the second version: sometimes, when a section starts on a new page, the hyperlink created by the "hyperref" package incorrectly points to the previous page, rather than to the page where the new section actually begins. In this particular case, we can add a "\clearpage" before the problematic section to resolve the problem.

% Third version:
\newcommand{\smartSectionClear}{%
	\needspace{3\baselineskip}
	\ifdim\dimexpr\pagegoal-\pagetotal\relax<3\baselineskip%
		\clearpage%
	\fi
}

% The third version of the "smartSectionClear" command combines the functionality of the "needspace" package with an additional check to potentially add a "\clearpage". 

% Automate the execution of the "\smartSectionClear" command:
% -----------------------------------------------------------

% The "\AddToHook" feature, which allows the automatic execution of the "\smartSectionClear" command before each instance of the "\section" command, was introduced in LaTeX from the 2021 version onwards. However, this template was developed using pdfTeX 3.14159265-2.6-1.40.18 (TeX Live 2017/Debian), which does not support that feature. We can achieve similar behavior using the "\pretocmd" command from the "etoolbox" package.

\pretocmd{\section}{\smartSectionClear}{}{}
% Pre-appends the "\smartSectionClear" command to the "\section" command.

\pretocmd{\subsection}{\smartSectionClear}{}{}
% Pre-appends the "\smartSectionClear" command to the "\subsection" command.

\pretocmd{\subsubsection}{\smartSectionClear}{}{}
% Pre-appends the "\smartSectionClear" command to the "\subsubsection" command.

%--------------------------------------------------------------

%--------------------SOURCES-FOR-COMMENTS----------------------

% The comments on LaTeX and its commands are based on the contents of https://latexref.xyz/, an unofficial reference manual for the LaTeX2e document preparation system. 

% Another reference used here is "LaTeXpedia" by Lorenzo Pantieri, 2021, http://www.lorenzopantieri.net/LaTeX_files/LaTeXpedia.pdf.

% The comments on the classes, styles or packages (and their commands and options) come from the description provided on CTAN (https://www.ctan.org/) and from the official documentation of the different classes, styles or packages.

%--------------------------------------------------------------

%--------------------------------------------------------------

%--------------------KEYWORDS----------------------------------

%--------------------------------------------------------------
%               Keyword helper commands
%--------------------------------------------------------------

%----------------------KEYWORDS---------------------------------

% The "\keyword{\<name>}{<text>}" command can be used to define a simple keyword: useful to never forget the "\xspace" at the end of the definition (without the "\xspace" every keywords inserted inside text should be followed or enclosed by curly brackets or else there would be no space added).

\newcommand{\keyword}[2]{%
	\newcommand{#1}{#2\xspace}%
}

% For testing
\newcommand{\simpleNewcommand}{newcommand}
\keyword{\keywordExample}{keyword}

%--------------------------------------------------------------

%----------------------EMPHASIZED-KEYWORDS---------------------

% The "\emphKeyword{\<name>}{<text>}" command can be used to define a keyword with emphasized text.

% Basic version.
%\newcommand{\emphKeyword}[2]{%
%	\newcommand{#1}{\emph{#2}\xspace}%
%}

% Improved version.
\newcommand{\emphKeyword}[2]{%
	\keyword{#1}{\emph{#2}}%
}

% For testing.
\emphKeyword{\emphKeywordExample}{emphasized-keyword}

%--------------------------------------------------------------

%----------------------TEXTTT-KEYWORDS-------------------------

% The "\ttKeyword{\<name>}{<text>}" command can be used to define a keyword with typewriter font (useful for code-related keywords).

% Basic version.
%\newcommand{\ttKeyword}[2]{%
%	\newcommand{#1}{\texttt{#2}\xspace}%
%}

% Improved version.
\newcommand{\ttKeyword}[2]{%
	\keyword{#1}{\texttt{#2}}%
}

% For testing.
\ttKeyword{\ttKeywordExample}{typewriter-keyword}

%--------------------------------------------------------------

%----------------------ACRONYM-KEYWORDS------------------------

% The "\acroKeyword{\<name>}{<acronym>}" command can be used to define a keywords for a defined acronym.

% Basic version.
%\newcommand{\acroKeyword}[2]{%
%	\newcommand{#1}{\emph{\ac{#2}}\xspace}%
%}

% Improved version.
\newcommand{\acroKeyword}[2]{%
	\emphKeyword{#1}{\ac{#2}}%
}

% For testing.
\acroKeyword{\AKE}{AKE}

%--------------------------------------------------------------

%----------------------INDEXED-KEYWORDS------------------------

% The "\idxKeyword{\<name>}{<text>}{<entry>[!<subentry>]}" command can be used to define a indexed keyword: every time it is used, it adds the specified entry in the analytical index.

% Basic version.
%\newcommand{\idxKeyword}[3]{%
%	\newcommand{#1}{#2\index{#3}\xspace}%
%}

% Improved version.
\newcommand{\idxKeyword}[3]{%
	\keyword{#1}{#2\index{#3}}%
}

% With "\index{<entry>}" you declare an entry in the index. With "\index{<entry>!<subentry>}" you declare a sub entry of the particular entry (it will be on the line below the entry and and indented). With "\index{<entry>!<subentry>!<subsubentry>}" you declare a sub entry of the particular sub entry.

% For testing.
\idxKeyword{\idxKeyOne}{first-indexed-keyword}{First idx keyword}
\idxKeyword{\idxKeyTwo}{second-indexed-keyword}{Second idx keyword}
\idxKeyword{\idxKeyThree}{third-indexed-keyword}{Third idx keyword}
\idxKeyword{\idxKeyTwoSub}{subitem}{Second idx keyword!Sub}
\idxKeyword{\idxKeyThreeSubsub}{subsubitem}{Third idx keyword!Other sub!Subsub}

%--------------------------------------------------------------

%-------------------INDEXED-EMPHASIZED-KEYWORDS----------------

% The "\idxEmphKeyword{\<name>}{<text>}{<entry>[!<subentry>]}" command can be used to define a indexed emphasized keyword.

% Basic version.
%\newcommand{\idxEmphKeyword}[3]{%
%	\newcommand{#1}{\emph{#2}\index{#3}\xspace}%
%}

% Improved version.
\newcommand{\idxEmphKeyword}[3]{%
	\idxKeyword{#1}{\emph{#2}}{#3}
}

% For testing.
\idxEmphKeyword{\idxEmphKeyOne}{first-idxemph-keyword}{First idxemph keyword}
\idxEmphKeyword{\idxEmphKeyTwo}{second-idxemph-keyword}{Second idxemph keyword}
\idxEmphKeyword{\idxEmphKeyThree}{third-idxemph-keyword}{Third idxemph keyword}
\idxEmphKeyword{\idxEmphKeyTwoSub}{emph-sub}{Second idxemph keyword!Emphsub}
\idxEmphKeyword{\idxEmphKeyThreeSubsub}{emph-subsub}{Third idxemph keyword!Other emphsub!Emphsubsub}

%--------------------------------------------------------------

%-------------------INDEXED-TEXTTT-KEYWORDS--------------------

% The "\idxTtKeyword{\<name>}{<text>}{<entry>[!<subentry>]}" command can be used to define a indexed keyword with typewriter font (useful for code-related keywords).

% Basic version.
%\newcommand{\idxTtKeyword}[3]{%
%	\newcommand{#1}{\texttt{#2}\index{#3}\xspace}%
%}

% Improved version.
\newcommand{\idxTtKeyword}[3]{%
	\idxKeyword{#1}{\texttt{#2}}{#3}
}

% For testing.
\idxTtKeyword{\idxTtKeyOne}{first-idxtt-keyword}{First idxtt keyword}
\idxTtKeyword{\idxTtKeyTwo}{second-idxtt-keyword}{Second idxtt keyword}
\idxTtKeyword{\idxTtKeyThree}{third-idxtt-keyword}{Third idxtt keyword}
\idxTtKeyword{\idxTtKeyTwoSub}{tt-sub}{Second idxtt keyword!Ttsub}
\idxTtKeyword{\idxTtKeyThreeSubsub}{tt-subsub}{Third idxtt keyword!Other ttsub!Ttsubsub}

%--------------------------------------------------------------

%--------------------INDEXED-ACRONYM-KEYWORDS------------------

% The "\idxAcroKeyword{\<name>}{<acronym>}{<entry>[!<subentry>]}" command can be used to define an indexed keyword for a defined acronym.

% Basic version.
%\newcommand{\idxAcroKeyword}[3]{%
%	\newcommand{#1}{\emph{\ac{#2}}\index{#3}\xspace}%
%}

% Improved version.
\newcommand{\idxAcroKeyword}[3]{%
	\idxEmphKeyword{#1}{\ac{#2}}{#3}%
}

% For testing.
\idxAcroKeyword{\IAKE}{IAKE}{I@\acs{IAKE}}
% By default the acronym short name will be added to the analytical index in the "Symbols" group and not in correct letter group; we can precede the index entry with the desired group followed by "@" to force the sorting.

%--------------------------------------------------------------

%-------------------AUTO-INDEXED-ACRONYM-KEYWORDS--------------

% The "\autoIdxAcroKeyword{\<name>}{<acronym>}{<group>}" command can be used to define an indexed keyword for a defined acronym without the need to specify the entry to use in the analytical index: the entry will be in the specified group and will correspond to the short name associated with the acronym.

% Basic version.
%\newcommand{\autoIdxAcroKeyword}[3]{%
%	\newcommand{#1}{\emph{\ac{#2}}\index{#3@\acs{#2}}\xspace}%
%}

% Improved version.
\newcommand{\autoIdxAcroKeyword}[3]{%
	\idxAcroKeyword{#1}{#2}{#3@\acs{#2}}%
}

% For testing.
\autoIdxAcroKeyword{\AIAKE}{AIAKE}{A}

%--------------------------------------------------------------

%------------------HYPERLINKS-HELPER-COMMANDS------------------

% The "\myHref{<URL>}{<text>" command is a wrapper of "\href" from the "hyperref" package (made the text specified as the second argument a hyperlink to the URL specified as the first argument); in my version the text will be rendered using the typewriter font.

\newcommand{\myHref}[2]{%
	\href{#1}{\texttt{#2}}%
}

% The "\https{<URL-without-protocol>}" command can be used to reference a "secure" web page.
\newcommand{\https}[1]{%
	\myHref{https://#1}{#1}%
}

% The "\http{<URL-without-protocol>}" command can be used to reference a "not secure" web page.
\newcommand{\http}[1]{%
	\myHref{http://#1}{#1}%
}

% The "\mailto{<email>}" command can be used to reference an email address.
\newcommand{\mailto}[1]{%
	\myHref{mailto://#1}{#1}%
}

%--------------------------------------------------------------

%----------------------SECURE-WEBPAGE-KEYWORDS-----------------

% The "\webpage{\<name>}{<URL-without-protocol>}" command can be used to define a keyword for a "secure" webpage.

% Basic version.
%\newcommand{\webpage}[2]{%
%	\newcommand{#1}{\href{https://#2}{\texttt{#2}}\xspace}
%}

% Improved version.
\newcommand{\webpage}[2]{%
	\keyword{#1}{\https{#2}}%
}

% For testing.
\webpage{\myWebpage}{francescomucci.github.io}

%--------------------------------------------------------------

%----------------------NOT-SECURE-WEBPAGE-KEYWORDS-------------

% The "\notsecurewebpage{\<name>}{<URL-without-protocol>}" command can be used to define a keyword for a "not secure" webpage.

% Basic version.
%\newcommand{\notsecwebpage}[2]{%
%	\newcommand{#1}{\href{http://#2}{\texttt{#2}}\xspace}
%}

% Improved version.
\newcommand{\notsecwebpage}[2]{%
	\keyword{#1}{\http{#2}}%
}

% For testing.
\notsecwebpage{\notSecureWebpageExample}{icetcs.ru.is}

%--------------------------------------------------------------

%----------------------MAIL-KEYWORDS---------------------------

% The "\mail{\<name>}{<email>}" command can be used to define a keyword for an email address.

% Basic version.
%\newcommand{\mail}[2]{%
%	\newcommand{#1}{\href{mailto://#2}{\texttt{#2}}\xspace}
%}

% Improved version.
\newcommand{\mail}[2]{%
	\keyword{#1}{\mailto{#2}}%
}

% For testing.
\mail{\myMail}{francesco.mucci@edu.unifi.it}

%--------------------------------------------------------------

%-------------------AUTOCAP-COMMAND----------------------------

% The "\autocap{<letter>}" command can be used on the initial letter of a keyword text to automatically capitalize the text after a full stop.

\newcommand{\autocap}[1]{%
	\ifnum%
		\ifhmode\spacefactor\else2000\fi>1005%
		\uppercase{#1}%
		\else{#1}%
	\fi%
}

% Since we use the use "\frenchspacing", a "\spacefactor" value greater than 1005 indicate the end of a sentence (the "\spacefactor" parameter affects how much stretching or shrinking can happen if the next thing in the input is a space). With "\ifhmode" we check if the current mode is horizontal mode, the mode typically used to make lines of text.

% For testing.
\keyword{\autocapKeywordExample}{\autocap{a}utocap-keyword}

% For testing.
\emphKeyword{\autocapEmphKeywordExample}{\autocap{a}utocap-emph-keyword}
% PROBLEM: unfortunately this command does not work correctly with emphasized keyword.

% For testing.
\ttKeyword{\autocapTtKeywordExample}{\autocap{a}utocap-tt-keyword}
% PROBLEM: unfortunately this command does not work correctly with keyword with typewriter font.

%--------------------------------------------------------------

%--------------------SOURCES-FOR-COMMENTS----------------------

% The comments on LaTeX and its commands are based on the contents of https://latexref.xyz/, an unofficial reference manual for the LaTeX2e document preparation system.

% The comments on the classes, styles or packages (and their commands and options) come from the description provided on CTAN (https://www.ctan.org/) and from the official documentation of the different classes, styles or packages.

% The comments on TeX conditional commands are also based on "TeX by Topic" (2017) by Victor Eijkhout.

%---------------------------------------------------------------

%--------------------------------------------------------------
%              Keywords for thesis title page
%--------------------------------------------------------------

\keyword{\myUni}{Università degli Studi di Firenze}
\keyword{\myFaculty}{Scuola di Scienze Matematiche, Fisiche e Naturali}
\keyword{\myDegreeLevel}{Laurea Magistrale}
\keyword{\myDegree}{Informatica}
\keyword{\myYear}{Anno Accademico 2023-2024}

\keyword{\myItalianTitle}{Progettazione di uno smart contract a supporto del protocollo di fair exchange di VeriOSS, una piattaforma bug bounty basata sulla blockchain}
\keyword{\myEnglishTitle}{Design of a smart contract to support the fair exchange protocol of VeriOSS, a blockchain-based bug-bounty platform}

\keyword{\myName}{Francesco Mucci}
\keyword{\myRelatore}{Rosario Pugliese}
\keyword{\myCorrelatore}{Gabriele Costa}
\keyword{\myOtherCorrelatore}{Letterio Galletta}

\keyword{\myVersione}{Versione 0.1.2}
\keyword{\myTime}{27 febbraio 2024}

%--------------------------------------------------------------
%		  Copyright and Creative Common Licenses
%--------------------------------------------------------------

%----------------------COPYRIGHT-------------------------------

% A copyright is a type of intellectual property that gives its owner the exclusive right to copy, distribute, adapt, display, and perform a creative work, usually for a limited time (source: https://en.wikipedia.org/wiki/Copyright).

%--------------------------------------------------------------

%--------------------COPYRIGHT-SIGN----------------------------

% The copyright sign is the symbol used in copyright notices for works; this symbol is widely recognized but, under the Berne Convention, is no longer required in most nations to assert a new copyright (source: https://en.wikipedia.org/wiki/Copyright_symbol).

\keyword{\myCopyright}{Copyright \textcopyright}

%--------------------------------------------------------------

%------------------COPYRIGHT-IN-ITALY--------------------------

% In Italy, when you create an original work, you are automatically protected without the need to add any kind of wording or symbol. The Italian copyright law assigns the author, at the time of creation of the work, two main rights: the moral right (diritto morale) and patrimonial rights (diritti patrimoniali).

% The moral right, which is inalienable and lasts forever, guarantees authorship of the work and its integrity: if I am the author, whatever rights I cede, the work will still remain created by me.

% Patrimonial rights allow the economic use of the work in any form and manner and provide rights of reproduction and distribution, communication to the public, translation and processing. Unlike moral right, patrimonial rights have a pre-established duration: they expire 70 years after the death of the author. Even anonymous works are protected: patrimonial rights last 70 years from the date of first publication. Patrimonial rights, unlike moral rights, can be transferred.

% Source for comments: https://www.wikimedia.it/news/diritto-di-autore-e-creative-commons-una-guida/.

%--------------------------------------------------------------

%--------------COPYRIGHT-VS-CREATIVE-COMMONS-------------------

% Creative Commons (CC) licenses are copyright licenses, inspired by copyleft, that clearly indicate which patrimonial rights the author keeps for himself and which he renounces. Through these licenses you retain the moral rights, so a CC license do not serve to protect the work or demonstrate authorship, but to make it available and usable by third parties. 

% These licenses act in addition to and on the basis of existing copyright: they do not present themselves as an alternative to copyright, but a license is applied to a work that is protected by copyright.

% Source for comments: https://www.wikimedia.it/news/diritto-di-autore-e-creative-commons-una-guida/.

%--------------------------------------------------------------

%------------CREATIVE-COMMONS-LICENSES-HELPER-COMMANDS---------

% The "\defCClicense{<key>}{<badge-name>}{<url-path>}{<text>}" command can be used to define a CC license.

\newcommand{\defCClicense}[4]{%
	\keyword{#1}{%
		\includegraphics[scale=0.80]{license-badge/pdf/#2} %
		\href{https://creativecommons.org/#3}{Creative Commons #4}%
	}%
}

% The "\defCCfourLicense{<key>}{<badge-name>}{<text>}" command can be used to define a CC 4.0 license.

\newcommand{\defCCfourLicense}[3]{%
	\defCClicense{#1}{#2}{licenses/#2/4.0/}{#3 4.0 International License}%
}

%--------------------------------------------------------------

%-----------------------CC-ZERO-1.0----------------------------

% CC Zero 1.0 Universal: by marking the work with a CC0 public domain dedication, the creator is giving up their copyright and allowing reusers to distribute, remix, adapt, and build upon the material in any medium or format, even for commercial purposes.

% Basic version.
%\newcommand{\CCZEROLicense}{\includegraphics[scale=0.80]{license-badge/pdf/zero} \href{https://creativecommons.org/publicdomain/zero/1.0/}{Creative Commons Zero 1.0 Universal}\xspace}

% Improved version.
\defCClicense{\CCZEROLicense}{zero}{publicdomain/zero/1.0/}{Zero 1.0 Universal}

%--------------------------------------------------------------

%-----------------------CC-BY-4.0------------------------------

% CC Attribution 4.0 International (CC BY 4.0): this license requires that reusers give credit to the creator. It allows reusers to distribute, remix, adapt, and build upon the material in any medium or format, even for commercial purposes.

% Basic version.
%\newcommand{\CCBYLicense}{\includegraphics[scale=0.80]{license-badge/pdf/by} \href{https://creativecommons.org/licenses/by/4.0/}{Creative Commons Attribution 4.0 International License}\xspace}

% Improved version.
\defCCfourLicense{\CCBYLicense}{by}{Attribution}

%--------------------------------------------------------------

%-----------------------CC-BY-SA-4.0---------------------------

% CC Attribution-ShareAlike 4.0 International (CC BY-SA 4.0): this license requires that reusers give credit to the creator. It allows reusers to distribute, remix, adapt, and build upon the material in any medium or format, even for commercial purposes. If others remix, adapt, or build upon the material, they must license the modified material under identical terms.

% Basic version.
%\newcommand{\CCBYSALicense}{\includegraphics[scale=0.80]{license-badge/pdf/by-sa} \href{https://creativecommons.org/licenses/by-sa/4.0/}{Creative Commons Attribution{\hyp}ShareAlike 4.0 International}\xspace}

% Improved version.
\defCCfourLicense{\CCBYSALicense}{by-sa}{Attribution{\hyp}ShareAlike}

%--------------------------------------------------------------

%-----------------------CC-BY-ND-4.0---------------------------

% CC Attribution-NoDerivatives 4.0 International (CC BY-ND 4.0): this license requires that reusers give credit to the creator. It allows reusers to copy and distribute the material in any medium or format in unadapted form only, even for commercial purposes.

% Basic version.
%\newcommand{\CCBYNDLicense}{\includegraphics[scale=0.80]{license-badge/pdf/by-nd} \href{https://creativecommons.org/licenses/by-nd/4.0/}{Creative Commons Attribution{\hyp}NoDerivatives 4.0 International}\xspace}

% Improved version.
\defCCfourLicense{\CCBYNDLicense}{by-nd}{Attribution{\hyp}NoDerivatives}

%--------------------------------------------------------------

%-----------------------CC-BY-NC-4.0---------------------------

% CC Attribution-NonCommercial 4.0 International (CC BY-NC 4.0): this license requires that reusers give credit to the creator. It allows reusers to distribute, remix, adapt, and build upon the material in any medium or format, for noncommercial purposes only.

% Basic version.
%\newcommand{\CCBYNCLicense}{\includegraphics[scale=0.80]{license-badge/pdf/by-nc} \href{https://creativecommons.org/licenses/by-nc/4.0/}{Creative Commons Attribution{\hyp}NonCommercial 4.0 International}\xspace}

% Improved version.
\defCCfourLicense{\CCBYNCLicense}{by-nc}{Attribution{\hyp}NonCommercial}

%--------------------------------------------------------------

%-----------------------CC-BY-NC-SA-4.0------------------------

% CC Attribution-NonCommercial-ShareAlike 4.0 International (CC BY-NC-SA 4.0): this license requires that reusers give credit to the creator. It allows reusers to distribute, remix, adapt, and build upon the material in any medium or format, for noncommercial purposes only. If others modify or adapt the material, they must license the modified material under identical terms.

% Basic version.
%\newcommand{\CCBYNCSALicense}{\includegraphics[scale=0.80]{license-badge/pdf/by-nc-sa} \href{https://creativecommons.org/licenses/by-nc-sa/4.0/}{Creative Commons Attribution{\hyp}NonCommercial{\hyp}ShareAlike 4.0 International}}

% Improved version.
\defCCfourLicense{\CCBYNCSALicense}{by-nc-sa}%
{Attribution{\hyp}NonCommercial{\hyp}ShareAlike}

%--------------------------------------------------------------

%-----------------------CC-BY-NC-ND-4.0------------------------

% CC Attribution-NonCommercial-NoDerivatives 4.0 International (CC BY-NC-ND 4.0): this license requires that reusers give credit to the creator. It allows reusers to copy and distribute the material in any medium or format in unadapted form and for noncommercial purposes only.

% Basic version.
%\newcommand{\CCBYNCNDLicense}{\includegraphics[scale=0.80]{license-badge/pdf/by-nc-nd} \href{https://creativecommons.org/licenses/by-nc-nd/4.0/}{Creative Commons Attribution{\hyp}NonCommercial{\hyp}NoDerivatives 4.0 International}\xspace}

% Improved version.
\defCCfourLicense{\CCBYNCNDLicense}{by-nc-nd}%
{Attribution{\hyp}NonCommercial{\hyp}NoDerivatives}

%--------------------------------------------------------------

%-----------------------CC-SUMMARY-----------------------------

% In summary: 
% - CC0: this work has been marked as dedicated to the public domain;
% - BY: credit must be given to you, the creator;
% - SA: adaptations must be shared under the same terms;
% - ND: no derivatives or adaptations of your work are permitted;
% - NC: only noncommercial use of your work is permitted.

% Source for comments: https://creativecommons.org/

%--------------------------------------------------------------

\newcommand{\myLicense}{\myCopyright}

%--------------------------------------------------------------

\input{keywords/keywords-preface}

%---------------------------------------------------------------
%              Math related keywords for thesis
%---------------------------------------------------------------

%---------------------MATH-KEYWORDS-HELPER-COMMANDS-------------

% The "\ensureMathKeyword{<name>}{<text>}" can be used to redefine commands that ordinarily can be used only in math mode, so that they can be used both in math and in plain text (using the command \ensuremath{<text>} we ensure that the text is typeset in math mode).

\newcommand{\ensureMathKeyword}[2]{%
	\keyword{#1}{\ensuremath{#2}}%
}

% The "\ensureMathbbKeyword{<name>}{<text>}" command can be used to define a math keyword, usable in plain text, with blackboard bold font (useful for set-related keywords).

\newcommand{\ensureMathBBKeyword}[2]{%
	\ensureMathKeyword{#1}{\mathbb{#2}}%
}

% NB: in math mode the "xspace" added at the end of text by "\keyword" does not work; when using this kind of keywords in math mode; so, in that mode, you need to add the required space manually using one of the following commands:
% - "\," for a thin space (1/6 of a quad);
% - "\:" for a medium space (2/9 of a quad);
% - "\;" for a thick space (5/18 of a quad);
% - "\ " for a space equivalent to a space in text mode;
% - "\quad" for a space that is equal to the width of the current font;
% - "\qquad" for a space double that of "\quad".

% The "\mathConjKeyword{<name>}{<conj>}" command is used to define a math keyword that will be preceded and followed by a thick space: therefore, it turns out that this command is useful for defining keywords that act as "conjunction" between mathematical formulas.

\newcommand{\mathConjKeyword}[2]{%
	\newcommand{#1}{\; #2 \;}%
}

% The "\mathTextConjKeyword{<name>}{<text-conj>}" command is used to define a math "conjuction" keyword with a plain text value.

\newcommand{\mathTextConjKeyword}[2]{%
	\mathConjKeyword{#1}{\text{#2}}%
}

%---------------------------------------------------------------

%---------------------SETS-OF-NUMBERS---------------------------

% Natural numbers.
%\newcommand{\N}{\ensuremath{\mathbb{N}}\xspace} 
\ensureMathBBKeyword{\N}{N} 

% Integer numbers.
%\newcommand{\Z}{\ensuremath{\mathbb{Z}}\xspace} 
\ensureMathBBKeyword{\Z}{Z}

% Rational numbers.
%\newcommand{\Q}{\ensuremath{\mathbb{Q}}\xspace} 
\ensureMathBBKeyword{\Q}{Q}

% Real numbers.
%\newcommand{\R}{\ensuremath{\mathbb{R}}\xspace} 
\ensureMathBBKeyword{\R}{R}

 % Complex numbers.
%\newcommand{\C}{\ensuremath{\mathbb{C}}\xspace}
\ensureMathBBKeyword{\C}{C}

%---------------------------------------------------------------

%--------------------CUSTOM-LOGICAL-OPERATORS-------------------

% Custom spaced logical AND.
%\newcommand{\myland}{\; \land \;} 
\mathConjKeyword{\myland}{\land}

% Custom spaced logical OR.
%\newcommand{\mylor}{\; \lor \;} 
\mathConjKeyword{\mylor}{\lor}

%---------------------------------------------------------------

%--------------------TEXT-CONJUCTION-KEYWORDS-------------------

%\newcommand{\e}{\; \text{e} \;}
\mathTextConjKeyword{\e}{e}

%\newcommand{\se}{\; \text{se} \;}
\mathTextConjKeyword{\se}{se}

%\newcommand{\con}{\; \text{con} \;}
\mathTextConjKeyword{\con}{con}

%\newcommand{\per}{\; \text{per} \;}
\mathTextConjKeyword{\per}{per}

%\newcommand{\allora}{\; \text{allora} \;}
\mathTextConjKeyword{\allora}{allora}

%\newcommand{\implica}{\; \text{implica} \;}
\mathTextConjKeyword{\implica}{implica}

%---------------------------------------------------------------

%---------------------CUSTOM-ARROWS-----------------------------

% Long two headed right arrow.
\newcommand{\longtwoheadrightarrow}{\relbar\joinrel\twoheadrightarrow}

% Long right arrow labelled with the passed argument.
\newcommand{\xlongrightarrow}[1]{\overset{#1}{\longrightarrow}}

% Dot right arrow with labelled with the passed argument.
\newcommand{\dotxrightarrow}[1]{\; \bullet \!\!\! \xrightarrow{#1}}

%---------------------------------------------------------------

%--------------------CUSTOM-PARENTHESIS-------------------------

% Wrap with double braket the argument: $\doublebracket{3}$ => [[3]].
\newcommand{\doublebracket}[1]{\llbracket #1 \rrbracket} 

% Tupla constructor with one argument: $\tupla{1,2,3}$ => <1,2,3>.
\newcommand{\tupla}[1]{\langle #1 \rangle} % 

%---------------------------------------------------------------

%-------------------OTHER-MATH-RELATED-NEW-COMMANDS-------------

% Tabulation for math environments.
\newcommand{\tab}{\hspace{2em}} 

% Custom spaced "vale che".
\newcommand{\valeche}{,\;}

% Custom spaced "tale che".
\newcommand{\tc}{:\;}

%---------------------------------------------------------------

%--------------------SOURCES-FOR-COMMENTS-----------------------

% The comments on LaTeX and its commands are based on the contents of https://latexref.xyz/, an unofficial reference manual for the LaTeX2e document preparation system.

% The comments on the classes, styles or packages (and their commands and options) come from the description provided on CTAN (https://www.ctan.org/) and from the official documentation of the different classes, styles or packages.

%----------------------------------------------------------------

%--------------------------------------------------------------
%                 Keywords for thesis body
%--------------------------------------------------------------

%----------------------KEYWORDS--------------------------------

% The "\keyword{\<name>}{<text>}" command can be used to define a simple keyword: useful to never forget the "\xspace" at the end of the definition (without the "\xspace" every keywords inserted inside text should be followed or enclosed by curly brackets or else there would be no space added).

\hrefKeyword{\HackerOne}{https://hackerone.com/bug-bounty-programs}{HackerOne}

\hrefKeyword{\Intigriti}{https://intigriti.com/companies/bug-bounty}{Intigriti}

\hrefKeyword{\Bugcrowd}{https://www.bugcrowd.com/products/bug-bounty/}{Bugcrowd}

\hrefKeyword{\Synack}{https://synack.com/solutions/go-beyond-bug-bounty/}{Synack}

\hrefKeyword{\Yogosha}{https://yogosha.com/bug-bounty/}{Yogosha}

\hrefKeyword{\GoogleBBP}{https://bughunters.google.com/}{Google}

\hrefKeyword{\MetaBBP}{https://facebook.com/whitehat}{Meta}

\hrefKeyword{\MicrosoftBBP}{https://microsoft.com/en-us/msrc/bounty}{Microsoft}

%--------------------------------------------------------------

%----------------------EMPHASIZED-KEYWORDS---------------------

% The "\emphKeyword{\<name>}{<text>}" command can be used to define a keyword with emphasized text.

%--------------------------------------------------------------

%----------------------TEXTTT-KEYWORDS-------------------------

% The "\ttKeyword{\<name>}{<text>}" command can be used to define a keyword with typewriter font (useful for code-related keywords).

%--------------------------------------------------------------

%----------------------ACRONYM-KEYWORDS------------------------

% The "\acroKeyword{\<name>}{<acronym>}" command can be used to define a keywords for a defined acronym.

%--------------------------------------------------------------

%----------------------INDEXED-KEYWORDS------------------------

% The "\idxKeyword{\<name>}{<text>}{<entry>[!<subentry>]}" command can be used to define a indexed keyword: every time it is used, it adds the specified entry in the analytical index.

%--------------------------------------------------------------

%----------------------INDEXED-EMPHASIZED-KEYWORDS-------------

% The "\idxEmphKeyword{\<name>}{<text>}{<entry>[!<subentry>]}" command can be used to define a indexed emphasized keyword.

\idxEmphKeyword{\Bug}{Bug}{Bug}
\idxEmphKeyword{\bug}{bug}{Bug}

\idxEmphKeyword{\Vulnerability}{Vulnerabilità}{Vulnerabilità}
\idxEmphKeyword{\vulnerability}{vulnerabilità}{Vulnerabilità}

\idxEmphKeyword{\BugBounty}{Bug Bounty}{Bug bounty}
\idxEmphKeyword{\Bugbounty}{Bug bounty}{Bug bounty}
\idxEmphKeyword{\bugbounty}{bug bounty}{Bounty}

\idxEmphKeyword{\BountyReward}{Bounty Reward}{Bounty reward}
\idxEmphKeyword{\Bountyreward}{Bounty reward}{Bounty reward}
\idxEmphKeyword{\bountyreward}{bounty reward}{Bounty reward}

\idxEmphKeyword{\BugReport}{Bug Report}{Bug Report}
\idxEmphKeyword{\Bugreport}{Bug report}{Bug report}
\idxEmphKeyword{\bugreport}{bug report}{Bug report}

\idxEmphKeyword{\InternalBBP}{Internal \acs{BBP}}{Internal \acs{BBP}}
\idxEmphKeyword{\internalBBP}{internal \acs{BBP}}{Internal \acs{BBP}}

\idxEmphKeyword{\BugBountyPlatform}{Bug Bounty Platform}{Bug bounty platform}
\idxEmphKeyword{\Bugbountyplatform}{Bug bounty platform}{Bug bounty platform}
\idxEmphKeyword{\bugbountyplatform}{bug bounty platform}{Bug bounty platform}

%--------------------------------------------------------------

%----------------------INDEXED-TEXTTT-KEYWORDS-----------------

% The "\idxTtKeyword{\<name>}{<text>}{<entry>[!<subentry>]}" command can be used to define a indexed keyword with typewriter font (useful for code-related keywords).

%--------------------------------------------------------------

%----------------------INDEXED-ACRONYM-KEYWORDS----------------

% The "\idxAcroKeyword{\<name>}{<acronym>}{<entry>[!<subentry>]}" command can be used to define an indexed keyword for a defined acronym.

%--------------------------------------------------------------

%----------------------AUTO-INDEXED-ACRONYM-KEYWORDS-----------

% The "\autoIdxAcroKeyword{\<name>}{<acronym>}{<group>}" command can be used to define an indexed keyword for a defined acronym without the need to specify the entry to use in the analytical index: the entry will be in the specified group and will correspond to the short name associated with the acronym.

\autoIdxAcroKeyword{\CVD}{CVD}{C}

\autoIdxAcroKeyword{\BBP}{BBP}{B}

\autoIdxAcroKeyword{\VDP}{VDP}{V}

\autoIdxAcroKeyword{\BI}{BI}{B}

\autoIdxAcroKeyword{\BH}{BH}{B}

%--------------------------------------------------------------

%----------------------SECURE-WEBPAGE-KEYWORDS-----------------

% The "\webpage{\<name>}{<URL-without-protocol>}" command can be used to define a keyword for a "secure" webpage.

%--------------------------------------------------------------

%----------------------NOT-SECURE-WEBPAGE-KEYWORDS-------------

% The "\notsecurewebpage{\<name>}{<URL-without-protocol>}" command can be used to define a keyword for a "not secure" webpage.

%--------------------------------------------------------------

%----------------------MAIL-KEYWORDS---------------------------

% The "\mail{\<name>}{<email>}" command can be used to define a keyword for an email address.

%--------------------------------------------------------------

%--------------------------------------------------------------

%--------------------SOURCES-FOR-COMMENTS----------------------

% The comments on LaTeX and its commands are based on the contents of https://latexref.xyz/, an unofficial reference manual for the LaTeX2e document preparation system.

% The comments on the classes, styles or packages (and their commands and options) come from the description provided on CTAN (https://www.ctan.org/) and from the official documentation of the different classes, styles or packages.

%---------------------------------------------------------------