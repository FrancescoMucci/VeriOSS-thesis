%--------------------------------------------------------------
%             Tables related packages for thesis   
%--------------------------------------------------------------

%--------------------USED-TABLES-PACKAGES----------------------

%--------------------------------------------------------------

%-------------TABLES-PACKAGES-LOADED-BY-CLASSICTHESIS----------

%\usepackage{tabularx} % WILL BE LOADED by "dia-classicthesis-ldpkg".
% The "tabularx" package defines the "tabularx" environment, an extension of "tabular" but, to set a table with the requested total width, modifies the widths of certain columns rather than the inter column space; the columns that may stretch are marked with the new token "X" in the preamble argument. This package requires the "array" package.

%\usepackage{booktabs} % WILL BE LOADED by "classicthesis".
% The "booktab" package enhances the quality of tables in LaTeX, providing extra commands as well as behind-the-scenes optimisation. From version 1.61, the package offers "longtable" compatibility.

%--------------------------------------------------------------

%--------------------UNUSED-TABLES-PACKAGES--------------------

%\usepackage{longtable} % LOAD BEFORE "hyperref".
% The "longtable" package allows you to write tables that continue to the next page. You can write captions within the table and headers and trailers for pages of table. Longtable arranges that the columns on successive pages have the same widths.

%\usepackage{multirow}
% The "multirow" package enable the construction of tables with cells that span more than one row of the table.

%---------------------------------------------------------------

%--------------------SOURCES-FOR-COMMENTS-----------------------

% The comments on LaTeX and its commands are based on the contents of https://latexref.xyz/, an unofficial reference manual for the LaTeX2e document preparation system.

% The comments on the classes, styles or packages (and their commands and options) come from the description provided on CTAN (https://www.ctan.org/) and from the official documentation of the different classes, styles or packages.

%---------------------------------------------------------------