\chapter{Preliminari}
\label{chap:preliminari}

\acresetall
% Empty the memory of the "\ac" macro: each time you use an acronym for the first time, the full name and the short name in brackets will be printed; afterwards only the acronym will be printed.

Gli obiettivi di questo capitolo sono i seguenti:
\begin{itemize}
\item dimostrare che lo studente possiede una comprensione completa dell'area di ricerca in cui opera \cite{pfandzelter2022thesis};
\item fornire al lettore le nozioni avanzate\footnotemark{} essenziali per la comprensione della tesi \cite{zobel2015writing};
\item inquadrare la tesi nel suo contesto, collegandola ai lavori precedenti, pubblicati o inediti, che costituiscono la base per il progetto svolto \cite{unibz2022thesis}.
\end{itemize}

\footnotetext{In questo contesto, consideriamo come conoscenza comune tutto ciò che è stato trattato nei corsi obbligatori dello specifico corso di laurea \cite{pfandzelter2022thesis}.}

Alla luce di quanto detto, sarà necessario:
\begin{itemize}
\item introdurre concetti, definizioni e terminologia che verranno utilizzati nel resto della tesi \cite{pfandzelter2022thesis};
\item presentare i lavori che costituiscono il fondamento del nostro progetto \cite{unibz2022thesis};
\item descrivere i metodi e le tecniche che formano la base del nostro lavoro \cite{fau2023thesis};
\item illustrare l'hardware e il software impiegato \cite{fau2023thesis}.
\end{itemize}

\section{Introduzione al capitolo}

All'inizio di ogni capitolo includeremo una breve introduzione che fornisce contesto al capitolo stesso; questa introduzione faciliterà la transizione logica da un capitolo all'altro mostrando come quello corrente si colleghi ai precedenti \cite{zobel2015writing}.

\medskip

Per essere più precisi, in ognuna di queste introduzioni forniremo \cite{unibz2022thesis}:
\begin{itemize}

\item gli obiettivi generali del capitolo, specificando cosa si intende affrontare e quale aspetto del nostro lavoro verrà esplorato;

\item una spiegazione di come il capitolo corrente si inserisca nel contesto più ampio della tesi, collegandolo ai temi generali e agli obiettivi più ampi del nostro lavoro;

\item un breve riassunto di come si è concluso il capitolo precedente e come quello corrente si costruisce sulle fondamenta del primo (ciò va fatto solo se pertinente);

\item un sommario del contenuto del capitolo, fornendo, ad esempio, una concisa panoramica delle sezioni e sottosezioni presenti.

\end{itemize}

\section{Nozioni preliminari}

Iniziamo con l'introduzione di concetti, definizioni e teorie fondamentali che costituiscono le basi del nostro lavoro. Questa sezione potrebbe includere, ad esempio, modelli matematici e teoremi essenziali per la comprensione della tesi.

\medskip

È importante tenere a mente che non stiamo scrivendo un libro di testo: ogni volta che introduciamo un termine o un concetto sarà sufficiente fornirne una breve spiegazione e includere un riferimento bibliografico; in tal modo il lettore potrà approfondire autonomamente l'argomento \cite{tuni2019guide}.

\medskip

Di seguito elenchiamo le fonti per le nozioni preliminari.

\begin{itemize}

\item Piattaforme bug bounty:
\begin{itemize}

\item \textit{"Supporting data-driven software development life-cycles with bug bounty programmes"} \cite{walshe2023bountythesis}

\item \textit{"Current State of Bug Bounty Programmes and Platforms"} \cite{walshe2023bountythesis3};

\item \textit{"An Empirical Study of Bug Bounty Programs"} \cite{walshe2020bountypaper};

\item \textit{"Bug hunters' perspectives on the challenges and benefits of the bug bounty ecosystem"} \cite{akgul2023bughunters};

\item \textit{"The Hackers' Viewpoint: Exploring Challenges and Benefits of Bug-Bounty Programs"} \cite{akgul2020bughunters};

\item \textit{"Bug Bounty Programs for Cybersecurity: Practices, Issues, and Recommendations"} \cite{malladi2020bugbounty};

\item \textit{"Web Science Challenges in Researching Bug Bounties"} \cite{fryer2017bugbounty}.

\end{itemize}

\item Blockchain, Ethereum e Solidity:
\begin{itemize}

\item \textit{"Mastering Ethereum: Building Smart Contracts and DApps"} \cite{antonopoulos2018mastering};

\item \textit{"Solidity Documentation - Release 0.8.18"} \cite{solidity0.8.18doc};

\item \textit{"Ethereum Development Documentation"} \cite{ethereum2024doc};

\end{itemize}

\item Verifica formale di smart contract Solidity:
\begin{itemize}

\item \textit{Formal Verification of Smart Contracts} \cite{ethereum2024scfv};

\item \textit{"A Survey of Smart Contract Formal Specification and Verification"} \cite{tolmach2021scfsvsurvey};

\item \textit{Formal Methods for the Verification of Smart Contracts: A Review} \cite{krichen2022scfmvreview};

\item \textit{"A Solicitous Approach to Smart Contract Verification"} \cite{otoni2023verification}.

\end{itemize}

\item Fair exchange protocol:
\begin{itemize}
\item TO DO
\end{itemize}

\end{itemize}


\section{Lavori precedenti}

Successivamente, se la tesi è una diretta continuazione di articoli o progetti di ricerca preesistenti, esporremo questi lavori evidenziando come la nostra ricerca si sviluppi a partire dalle loro fondamenta.

\medskip

Elenco dei lavori precedenti:

\begin{itemize}

\item \textit{"VeriOSS: Using the Blockchain to Foster Bug Bounty Programs"} \cite{canidio2021verioss};

\item \textit{"Verifying a Blockchain-Based Remote Debugging Protocol for Bug Bounty"} \cite{degano2021verioss};

\end{itemize}


\section{Metodi e tecniche utilizzate}

In questa sezione descriveremo i metodi e le tecniche adottate nel corso della tesi, quali le analisi statistiche o i metodi di sviluppo software utilizzati per portare avanti il nostro progetto.

\medskip

Nel caso di un progetto implementativo, descriveremo la metodologia di sviluppo impiegata (es. Agile, Waterfall, TDD, BDD) e come questa abbia influenzato il processo di implementazione.

\section{Tecnologie utilizzate}

Concludiamo il capitolo presentando le tecnologie, inclusi gli strumenti software e hardware, impiegati nella nostra ricerca, spiegando il loro ruolo e come hanno contribuito al raggiungimento degli obiettivi della tesi.

\medskip

Nel caso di un progetto implementativo, in questa sezione elencheremo i linguaggi di programmazione, i framework, i database e altri strumenti utilizzati per lo sviluppo e il test del software. Ogni scelta andrà motivata illustrando quali sono i vantaggi per il nostro progetto.

\section{Riassunto del capitolo e conclusioni}

Alla fine di ogni capitolo includeremo un breve riassunto del suo contenuto, una riflessione su come quanto trattato contribuisca agli obiettivi generali della tesi e, per concludere, un'anticipazione di come i capitoli successivi faranno uso di quanto introdotto in quello corrente (in tal modo metteremo in evidenza come questi sono tra loro collegati) \cite{zobel2015writing}.