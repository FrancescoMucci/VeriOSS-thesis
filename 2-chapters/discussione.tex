\chapter{Discussione}
\label{chap:discussione}

\acresetall
% Empty the memory of the "\ac" macro: each time you use an acronym for the first time, the full name and the short name in brackets will be printed; afterwards only the acronym will be printed.

Questo capitolo si focalizza sull'analisi critica e obiettiva del lavoro svolto, esaminando sia l'approccio risolutivo al problema affrontato sia il metodo di valutazione impiegato per dimostrarne l'efficacia \cite{pfandzelter2022thesis}. 

\medskip

Per condurre questa analisi possiamo lasciarci guidare dalle successive domande  \cite{zobel2015writing}:
\begin{itemize}
\item In cosa il lavoro ha avuto successo?
\item E in cosa invece ha fallito?
\item Quali problemi non sono stati risolti?
\item Quali nuove domande sono emerse?
\item Quali approcci alternativi avremmo potuto considerare?
\item Quali sono le implicazioni dei risultati ottenuti?
\end{itemize}

\medskip

Altre domande che possono venirci in aiuto sono le seguenti \cite{tuni2019guide}:
\begin{itemize}
\item I risultati ottenuti corrispondono agli obiettivi iniziali? 
\item Siamo riusciti a rispondere alle nostre domande di ricerca?
\item Qual'è l'importanza scientifica e pratica dei risultati ottenuti?
\end{itemize}

\medskip

Inizieremo, quindi, l'analisi valutando se i risultati ottenuti corrispondono agli obiettivi prefissati della nostra ricerca e se abbiamo efficacemente risposto alle domande di ricerca poste. Successivamente, valuteremo le debolezze e le limitazioni del nostro lavoro, esplorando anche le questioni ancora aperte, le nuove domande emerse e gli eventuali approcci alternativi che avremmo potuto esplorare. Concluderemo discutendo l'importanza scientifica e pratica dei nostri risultati, evidenziando come questi possano essere applicati nella pratica e contribuire alle teorie esistenti.

\medskip

Se ritenuto necessario, questo capitolo può essere convertito in una sezione del capitolo conclusivo; tale sezione andrebbe posizionata subito dopo il riassunto della tesi \cite{zobel2015writing}.

\section{Introduzione al capitolo}

All'inizio di ogni capitolo includeremo una breve introduzione che fornisce contesto al capitolo stesso; questa introduzione faciliterà la transizione logica da un capitolo all'altro mostrando come quello corrente si colleghi ai precedenti \cite{zobel2015writing}.

\medskip

Per essere più precisi, in ognuna di queste introduzioni forniremo \cite{unibz2022thesis}:
\begin{itemize}

\item gli obiettivi generali del capitolo, specificando cosa si intende affrontare e quale aspetto del nostro lavoro verrà esplorato;

\item una spiegazione di come il capitolo corrente si inserisca nel contesto più ampio della tesi, collegandolo ai temi generali e agli obiettivi più ampi del nostro lavoro;

\item un breve riassunto di come si è concluso il capitolo precedente e come quello corrente si costruisce sulle fondamenta del primo (ciò va fatto solo se pertinente);

\item un sommario del contenuto del capitolo, fornendo, ad esempio, una concisa panoramica delle sezioni e sottosezioni presenti.

\end{itemize}

\section{Obiettivi raggiunti}

In questa sezione valuteremo se i risultati ottenuti corrispondono agli obiettivi prefissati e se siamo riusciti a rispondere alle nostre domande di ricerca. 

\medskip

Non sarà necessario illustrare nuovamente i risultati in quanto questi dovrebbero già essere stati trattati nel capitolo \textit{Valutazione} \cite{pfandzelter2022thesis}.

\section{Debolezze e limitazioni}

Valutiamo le debolezze e le limitazioni del nostro lavoro.

\section{Questioni irrisolte}

Illustriamo eventuali questioni irrisolte.

\section{Nuove domande emerse}

Evidenziamo le nuove domande che sono emerse dalla ricerca svolta.

\section{Approcci alternativi}

Trattiamo approcci alternativi che avremmo potuto prendere in considerazione.

\section{Impatto scientifico e pratico dei risultati}

Discutiamo infine dell'importanza scientifica e pratica dei risultati ottenuti, riflettendo sulle loro potenziali implicazioni e ripercussioni. Considereremo, quindi, come questi risultati possano essere applicati nella pratica o contribuire a teorie esistenti.

\medskip

Per scrivere questa sezione, possiamo provare a rispondere alla seguente domanda: "Quali lezioni o scoperte fatte durante il nostro lavoro di ricerca possono essere applicate in altri contesti?" \cite{mannisto2022guide}.

\section{Riassunto del capitolo e conclusioni}

Alla fine di ogni capitolo includeremo un breve riassunto del suo contenuto, una riflessione su come quanto trattato contribuisca agli obiettivi generali della tesi e, per concludere, un'anticipazione di come i capitoli successivi faranno uso di quanto introdotto in quello corrente (in tal modo metteremo in evidenza come questi sono tra loro collegati) \cite{zobel2015writing}.