\chapter{Valutazione}
\label{chap:valutazione}

\acresetall
% Empty the memory of the "\ac" macro: each time you use an acronym for the first time, the full name and the short name in brackets will be printed; afterwards only the acronym will be printed.

%----------SUGGERIMENTI-SUL-CONTENUTO----------%
%Questo è il secondo dei due capitoli centrali in cui descriviamo il lavoro progettuale svolto. Il suo obiettivo principale è dimostrare, attraverso un processo di valutazione, l'efficacia dell'approccio risolutivo presentato nel capitolo precedente \cite{pfandzelter2022thesis}.
%
%\medskip
%
%Il metodo di valutazione scelto dipenderà dalla specifica domanda di ricerca; alcuni metodi che potremmo adottare sono i seguenti \cite{pfandzelter2022thesis}:
%\begin{itemize}
%
%\item \textbf{Implementazione}: nel caso in cui abbiamo progettato l'architettura di un nuovo sistema, possiamo dimostrarne la validità fornendo una sua implementazione; in tal caso, è essenziale che questa sia accompagnata da dei test che ne verifichino il comportamento o da dei benchmark che ne attestino il miglioramento rispetto a soluzioni esistenti.
%
%\item \textbf{Verifica formale}: forniamo una dimostrazione formale della correttezza del nostro approccio.
%
%\item \textbf{Simulazione}: testare la soluzione proposta in un ambiente controllato può essere un metodo efficace per valutarla; per esempio, potremmo creare un ambiente simulato per eseguire il nostro algoritmo o sistema; se scegliamo questa via, sarà cruciale definire accuratamente cosa misurare durante la simulazione al fine di ottenere dei risultati significativi.
%
%\end{itemize}
%
%In sostanza, a seconda del metodo di valutazione scelto, dovremo presentare: i risultati sperimentali; i teoremi e le relative dimostrazioni; l'analisi dei dati e le scoperte fatte  \cite{zobel2015writing}.
%
%\medskip
%
%La struttura di questo capitolo, esattamente come quella del precedente, dipenderà dal tipo di progetto intrapreso e, in genere, tratterà le fasi conclusive del suo sviluppo \cite{unibz2022thesis}: in un progetto implementativo, presenteremo una discussione dettagliata dell'implementazione e delle sue criticità, includendo anche una descrizione dei test eseguiti.
%----------------------------------------------%

\section{Introduzione al capitolo}

%----------SUGGERIMENTI-SUL-CONTENUTO----------%
%All'inizio di ogni capitolo includeremo una breve introduzione che fornisce contesto al capitolo stesso; questa introduzione faciliterà la transizione logica da un capitolo all'altro mostrando come quello corrente si colleghi ai precedenti \cite{zobel2015writing}.
%
%\medskip
%
%Per essere più precisi, in ognuna di queste introduzioni forniremo \cite{unibz2022thesis}:
%\begin{itemize}
%
%\item gli obiettivi generali del capitolo, specificando cosa si intende affrontare e quale aspetto del nostro lavoro verrà esplorato;
%
%\item una spiegazione di come il capitolo corrente si inserisca nel contesto più ampio della tesi, collegandolo ai temi generali e agli obiettivi più ampi del nostro lavoro;
%
%\item un breve riassunto di come si è concluso il capitolo precedente e come quello corrente si costruisce sulle fondamenta del primo (ciò va fatto solo se pertinente);
%
%\item un sommario del contenuto del capitolo, fornendo, ad esempio, una concisa panoramica delle sezioni e sottosezioni presenti.
%
%\end{itemize}
%----------------------------------------------%

\section{Implementazione}

%----------SUGGERIMENTI-SUL-CONTENUTO----------%
%In questa sezione presenteremo un'analisi dettagliata dell'implementazione del sistema illustrato nel capitolo precedente.
%----------------------------------------------%

\subsection{Implementazione componente 1}

%----------SUGGERIMENTI-SUL-CONTENUTO----------%
%Per ogni componente implementata, forniremo, anzitutto, una descrizione accurata delle sue sotto-componenti e delle risorse che usa o gestisce; dopodiché, spiegheremo come essa esegua i compiti necessari per adempiere alle proprie responsabilità.
%----------------------------------------------%

\subsection{Implementazione componente n}

\subsection{Sfide implementative e soluzioni}

%----------SUGGERIMENTI-SUL-CONTENUTO----------%
%Illustreremo eventuali sfide incontrate durante il processo di implementazione e come queste siano state superate.
%----------------------------------------------%

\section{Test}

%----------SUGGERIMENTI-SUL-CONTENUTO----------%
%In questa sezione descriviamo i test effettuati per verificare che il sistema si comporti nel modo atteso.
%----------------------------------------------%

\subsection{Test d'unit\`a}

%----------SUGGERIMENTI-SUL-CONTENUTO----------%
%L'obiettivo dei test d'unità è verificare che le singole componenti funzionino correttamente quando isolate dal resto del sistema \cite{bettini2021tdd}.
%----------------------------------------------%

\subsubsection{Test componente 1}

\subsubsection{Test componente n}

\subsection{Test d'integrazione}

%----------SUGGERIMENTI-SUL-CONTENUTO----------%
%L'obiettivo dei test d'integrazione è verificare che due o più componenti del sistema, già testate individualmente in isolamento, continuino a funzionare correttamente una volta messe insieme e fatte interagire \cite{bettini2021tdd}.
%----------------------------------------------%

\subsubsection{Test integrazione componenti 1 e 2}

\subsubsection{Test integrazione componenti n-1 e n}

\subsection{Test end-to-end}

%----------SUGGERIMENTI-SUL-CONTENUTO----------%
%L'obiettivo dei test end-to-end è verificare che il nostro sistema operi correttamente durante l'interazione con un suo client, sia esso umano o un'altra applicazione, attraverso le interfacce fornite \cite{bettini2021tdd}.
%----------------------------------------------%

\section{Qualit\`a dei test}

%----------SUGGERIMENTI-SUL-CONTENUTO----------%
%Descriviamo i metodi e le tecniche impiegati per valutare la qualità dei nostri test.
%----------------------------------------------%

\subsection{Test coverage}

%----------SUGGERIMENTI-SUL-CONTENUTO----------%
%Illustreremo la metodologia adottata per monitorare il test coverage, ovvero la percentuale di linee di codice di ogni singola componente eseguite durante i test. Più alto sarà il test coverage e più bassa sarà la probabilità che la componente contenga dei bug non rilevati dai test \cite{bettini2021tdd}.
%
%\medskip
%
%È buona norma puntare a raggiungere un test coverage del 100\%, utilizzando unicamente i test d'unità per la specifica componente; tuttavia, un coverage completo non garantisce di per sé la correttezza dei test (è condizione necessaria, ma non sufficiente): ci dice solamente che tutte le linee di codice sono state eseguite, ma non se le asserzioni dei test sono esaustive e appropriate \cite{bettini2021tdd}.
%----------------------------------------------%

\subsection{Mutation testing}

%----------SUGGERIMENTI-SUL-CONTENUTO----------%
%Per aumentare la fiducia nella qualità dei nostri test, possiamo utilizzare un framework di mutation testing per introdurre mutazioni nelle componenti testate e verificare che i nostri test d'unità rilevino queste mutazioni (fallendo) \cite{bettini2021tdd}.
%
%\medskip
%
%Dato che il mutation testing è un processo che richiede tempo, di solito viene utilizzato per valutare unicamente i test d'unità delle componenti che contengono la logica del sistema \cite{bettini2021tdd}.
%----------------------------------------------%

\section{Risultati}

%----------SUGGERIMENTI-SUL-CONTENUTO----------%
%In questa sezione illustriamo in modo chiaro e conciso quali sono i risultati ottenuti dalla nostra ricerca, evitando, almeno per il momento, qualsiasi tipologia di analisi critica \cite{mannisto2022guide}.
%
%\medskip
%
%Nel contesto di un progetto implementativo, presenteremo, anzitutto, una sintesi dei risultati ottenuti dai test. Dopodiché, illustreremo come l'implementazione realizzata soddisfi i requisiti specificati nel capitolo precedente e evidenzieremo i risultati chiave relativi alle prestazioni, alla scalabilità e alla manutenibilità del sistema.
%----------------------------------------------%

\section{Riassunto del capitolo e conclusioni}

%----------SUGGERIMENTI-SUL-CONTENUTO----------%
%Alla fine di ogni capitolo includeremo un breve riassunto del suo contenuto, una riflessione su come quanto trattato contribuisca agli obiettivi generali della tesi e, per concludere, un'anticipazione di come i capitoli successivi faranno uso di quanto introdotto in quello corrente (in tal modo metteremo in evidenza come questi sono tra loro collegati) \cite{zobel2015writing}.
%----------------------------------------------%