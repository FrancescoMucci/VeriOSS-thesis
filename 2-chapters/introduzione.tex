\chapter{Introduzione} 
\label{chap:introduzione}

\acresetall
% Empty the memory of the "\ac" macro: each time you use an acronym for the first time, the full name and the short name in brackets will be printed; afterwards only the acronym will be printed.

Lo scopo principale di questo capitolo è presentare il problema di ricerca trattato, illustrare gli obiettivi della tesi e delineare una panoramica del suo contenuto \cite{pfandzelter2022thesis}.

\medskip

L'introduzione, insieme alle conclusioni, rappresenta uno dei capitoli più importanti dell'intera tesi: molti lettori si concentreranno principalmente su questi due e daranno un'occhiata rapida alle figure e alle tabelle presenti nel resto del lavoro \cite{tuni2019guide}.

\section{Contesto}

Per prima cosa, come indicato sia da Pfandzelter \etAl \cite{pfandzelter2022thesis} che da Zobel \cite{zobel2015writing}, presenteremo l'area di ricerca che fa da sfondo al nostro lavoro di tesi. In questa sezione forniremo dunque una breve panoramica del particolare campo di studio, evidenziandone la rilevanza e le motivazioni per cui merita attenzione.

\section{Problema affrontato}

In secondo luogo, introdurremo lo specifico problema di ricerca che si vuole affrontare e, anche in questo caso, argomenteremo al fine di mettere in evidenza la sua importanza \cite{zobel2015writing}.

\section{Stato dell'arte}

Successivamente, come suggerito da Zobel \cite{zobel2015writing}, riassumeremo in modo sintetico le soluzioni standard al problema affrontato, enfatizzando le limitazioni di queste soluzioni per far comprendere meglio le motivazioni dietro al nostro lavoro di ricerca.

\section{Domande di ricerca}

In questa sezione elencheremo esplicitamente le domande di ricerca che guideranno il nostro studio, cioè le domande che identificano le lacune nelle conoscenze esistenti che cercheremo di colmare \cite{pfandzelter2022thesis}. Ad ogni domanda identificata sarà associato un obiettivo che la nostra tesi si prefigge di raggiungere; tali obiettivi indicano cosa abbiamo intenzione di dimostrare, sviluppare, testare o esplorare.

\medskip

Sarà appropriato introdurre le domande di ricerca nell'introduzione solo se risultano chiare anche senza aver prima esaminato i lavori precedenti; qualora ciò non fosse possibile, potrebbe essere opportuno posticipare la loro presentazione \cite{mannisto2022guide}.

\section{Approccio usato}

A questo punto, specificheremo l'approccio seguito per risolvere il problema affrontato, fornendo una breve descrizione della soluzione proposta. Zobel \cite{zobel2015writing} ci ricorda di menzionare eventuali articoli usati come base di partenza per il lavoro svolto e di evidenziare le motivazioni per cui la soluzione fornita può essere ritenuta efficace, trattando in modo succinto le tecniche usate per valutarla.

\medskip

È in questa sezione che specifichiamo come abbiamo risposto alle domande di ricerca illustrando brevemente le tecniche, i processi e i metodi utilizzati per raggiungere i nostri obiettivi \cite{pfandzelter2022thesis}.

\section{Contributi originali}

In questa penultima sezione, metteremo in evidenza i contributi originali dati dal nostro lavoro di tesi alla specifica area di ricerca \cite{pfandzelter2022thesis}: andremo quindi a illustrare sinteticamente i nuovi metodi, le teorie, i modelli o le implementazioni software introdotte dal nostro lavoro.

\medskip

Se il lavoro presentato è parte di un progetto più grande, è importante specificare chiaramente quale sia stato il nostro particolare contributo a tale progetto \cite{tuni2019guide}.

\section{Struttura della tesi}

Per concludere, delineeremo la struttura della tesi: per ogni capitolo forniremo una breve descrizione del suo contenuto e del suo contributo al lavoro complessivo \cite{tuni2019guide}.

\medskip

Seguendo l'approccio proposto da Pfandzelter \etAl \cite{pfandzelter2022thesis}, il restante lavoro di tesi è strutturato nei seguenti capitoli:

\begin{itemize}

\item \Cref{chap:preliminari} - \textsc{preliminari}: introduce le nozioni fondamentali per la comprensione del lavoro svolto e inquadra la tesi collegandola a eventuali lavori precedenti.

\item \Cref{chap:approccio} - \textsc{approccio}: illustra l'idea risolutiva proposta per il problema affrontato.

\item \Cref{chap:valutazione} - \textsc{valutazione}: dimostra, attraverso un processo di valutazione, l'efficacia dell'approccio risolutivo adottato.

\item \Cref{chap:discussione} - \textsc{discussione}: conduce un'analisi critica e oggettiva dell'approccio e del metodo di valutazione utilizzati.

\item \Cref{chap:correlati} - \textsc{lavori correlati}: esamina le ricerche correlate mettendole a confronto con il lavoro realizzato.

\item \Cref{chap:conclusioni} - \textsc{conclusioni}: riassume in modo conciso quanto svolto (concentrandosi principalmente sui risultati ottenuti) e suggerisce possibili sviluppi futuri.

\end{itemize}

Oltre ai capitoli principali, potremmo decidere di includere nella nostra tesi delle appendici; 
queste dovrebbero contenere materiale che, pur essendo necessario per comprendere il lavoro svolto, non rappresenta una parte centrale della tesi o non può essere inserito nel testo principale a causa delle sue dimensioni eccessive o del formato particolare \cite{tuni2019guide}.

\medskip

Ad esempio, potremmo inserire come appendice: codici sorgente svilupati; dimostrazioni matematiche articolate; tabelle e figure di grandi dimensioni; diagrammi complessi; descrizioni dettagliate di processi di misura, degli esperimenti condotti e dei risultati sperimentali ottenuti; manuali d'uso e di manutenzione redatti; sondaggi condotti con relativi risultati \cite{tuni2019guide}.

\medskip

La tesi include, oltre ai capitoli, anche le seguenti appendici:

\begin{itemize}

\item \Cref{appendix:codici} - \textsc{codici sorgente addizionali}:

\item \Cref{appendix:dimostrazioni} - \textsc{dimostrazioni addizionali}:

\end{itemize}