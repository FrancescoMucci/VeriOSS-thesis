\chapter{Conclusioni}
\label{chap:conclusioni}

\acresetall
% Empty the memory of the "\ac" macro: each time you use an acronym for the first time, the full name and the short name in brackets will be printed; afterwards only the acronym will be printed.

Il capitolo conclusivo, insieme all'introduzione, rappresenta uno dei capitoli più importanti dell'intera tesi: molti lettori si concentreranno principalmente su questi due e daranno un'occhiata rapida alle figure e alle tabelle presenti nel resto del lavoro \cite{tuni2019guide}.

\section{Riassunto della tesi}

Il capitolo conclusivo della tesi si apre comunemente con un riassunto sintetico del lavoro svolto che si sofferma soprattutto sui risultati chiave raggiunti \cite{fau2023thesis}. Questa sezione dovrà quindi contenere:
\begin{enumerate}

\item un riepilogo del problema indagato \cite{pfandzelter2022thesis};

\item una descrizione essenziale del metodo adottato per risolverlo \cite{pfandzelter2022thesis};

\item una sintesi dei risultati conseguiti \cite{zobel2015writing} e del contributo fornito dal nostro lavoro di tesi \cite{aceto2023paper}.

\end{enumerate}

\section{Sviluppi futuri}

È essenziale dedicare una sezione alla prospettiva di ulteriori ricerche che possano estendere o approfondire lo studio presente. Questa parte dovrebbe concentrarsi principalmente sugli aspetti ancora da esplorare, piuttosto che sulle metodologie da adottare \cite{fau2023thesis}. Potrebbe includere, per esempio, proposte non implementate in questa tesi o strategie per superare le limitazioni e le debolezze individuate \cite{pfandzelter2022thesis}.