\chapter{Lavori correlati}
\label{chap:correlati}

\acresetall
% Empty the memory of the "\ac" macro: each time you use an acronym for the first time, the full name and the short name in brackets will be printed; afterwards only the acronym will be printed.

L'obiettivo di questo capitolo è di sottolineare l'originalità e la rilevanza del nostro lavoro di tesi rispondendo alle seguenti domande chiave \cite{aceto2023paper}:
\begin{itemize}
\item Quali sono le origini delle nostre idee?
\item Sono state pubblicate o proposte idee simili precedentemente?
\item Quali sono gli aspetti originali del nostro lavoro?
\end{itemize}

In sostanza, il capitolo serve a mostrare che non ci siamo limitati a reinventare la ruota \cite{tuni2019guide}. Per far ciò, esamineremo altri studi presenti in letteratura che cercano di risolvere lo stesso problema di ricerca o uno correlato \cite{pfandzelter2022thesis}, mettendo in evidenza le similitudini e le differenze rispetto al nostro approccio \cite{tuni2019guide}.

\medskip

Ogni lavoro correlato verrà affrontato separatamente seguendo questa metodologia: \cite{pfandzelter2022thesis}:
\begin{enumerate}
\item riassumiamo l'idea principale dello studio;
\item analizziamo i punti di forza, le limitazioni e i difetti (anche in confronto con il nostro approccio);
\item evidenziamo se e come il nostro studio sia stato influenzato dal lavoro esaminato.
\end{enumerate}

Tireremo le fila dell'analisi evidenziando se emergono delle tendenze e identificando sia l'approccio più diffuso sia quello raccomandato per affrontare il problema di ricerca in questione. Per concludere, sottolineeremo le nuove conoscenze prodotte dal nostro lavoro di ricerca \cite{tuni2019guide}.

\medskip

Sebbene questo capitolo possa essere posizionato sia dopo i preliminari teorici e tecnici sia prima delle conclusioni, seguendo il consiglio di Pfandzelter \etAl \cite{pfandzelter2022thesis}, la seconda opzione è preferibile per evitare di presentare i lavori correlati prima che il lettore abbia acquisito una piena comprensione dell'approccio utilizzato nel nostro lavoro di tesi. Alternativamente, se l'analisi dei lavori correlati è sufficientemente breve, può essere direttamente integrata nel capitolo sui preliminari o in quello contenente la discussione \cite{mannisto2022guide}.

\section{Introduzione al capitolo}

All'inizio di ogni capitolo includeremo una breve introduzione che fornisce contesto al capitolo stesso; questa introduzione faciliterà la transizione logica da un capitolo all'altro mostrando come quello corrente si colleghi ai precedenti \cite{zobel2015writing}.

\medskip

Per essere più precisi, in ognuna di queste introduzioni forniremo \cite{unibz2022thesis}:
\begin{itemize}

\item gli obiettivi generali del capitolo, specificando cosa si intende affrontare e quale aspetto del nostro lavoro verrà esplorato;

\item una spiegazione di come il capitolo corrente si inserisca nel contesto più ampio della tesi, collegandolo ai temi generali e agli obiettivi più ampi del nostro lavoro;

\item un breve riassunto di come si è concluso il capitolo precedente e come quello corrente si costruisce sulle fondamenta del primo (ciò va fatto solo se pertinente);

\item un sommario del contenuto del capitolo, fornendo, ad esempio, una concisa panoramica delle sezioni e sottosezioni presenti.

\end{itemize}

\section{Panoramica sullo stato dell'arte}

In questa sezione descriviamo i progressi e le scoperte principali compiute fino ad oggi nel campo di ricerca pertinente al nostro lavoro, stabilendo così il contesto per l'analisi dei lavori correlati.

\section{Lavori debolmente correlati}

Procediamo con un'analisi dei lavori che hanno un legame indiretto o generale con il problema di ricerca che stiamo affrontando.

\subsection{Lavoro debolmente correlato 1}
\subsubsection{Idea principale}
\subsubsection{Punti di forza}
\subsubsection{Limitazioni e difetti}
\subsubsection{Influenza sul nostro lavoro}

\subsection{Lavoro debolmente correlato 2}
\subsubsection{Idea principale}
\subsubsection{Punti di forza}
\subsubsection{Limitazioni e difetti}
\subsubsection{Influenza sul nostro lavoro}

\section{Lavori strettamente correlati}

Esaminiamo poi in dettaglio i lavori che si propongono di risolvere il problema di ricerca da noi affrontato e che adottano un approccio simile al nostro.

\subsection{Lavoro strettamente correlato 1}
\subsubsection{Idea principale}
\subsubsection{Punti di forza}
\subsubsection{Limitazioni e difetti}
\subsubsection{Influenza sul nostro lavoro}

\subsection{Lavoro strettamente correlato 2}
\subsubsection{Idea principale}
\subsubsection{Punti di forza}
\subsubsection{Limitazioni e difetti}
\subsubsection{Influenza sul nostro lavoro}

\section{Tendenze identificate}

Evidenziamo le eventuali tendenze emerse a seguito dell'analisi dei lavori correlati e identifichiamo sia l'approccio più diffuso sia quello raccomandato per affrontare il particolare problema di ricerca.

\section{Lacune nella letteratura e nostro contributo}

Identifichiamo le lacune nella letteratura esistente e mostriamo come il nostro lavoro miri a colmarle, mettendo in tal modo in evidenza il contributo originale apportato dalla nostra ricerca.

\section{Riassunto del capitolo e conclusioni}

Alla fine di ogni capitolo includeremo un breve riassunto del suo contenuto, una riflessione su come quanto trattato contribuisca agli obiettivi generali della tesi e, per concludere, un'anticipazione di come i capitoli successivi faranno uso di quanto introdotto in quello corrente (in tal modo metteremo in evidenza come questi sono tra loro collegati) \cite{zobel2015writing}.