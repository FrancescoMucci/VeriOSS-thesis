\chapter*{Prefazione}
% The starred variants of all sectioning commands (\chapter*, \section*, ...) produce unnumbered headings which do not appear in the table of contents or in the page header. The absence of a running head often has an unwanted side effect: if, for example, a chapter set using \chapter* spans several pages, then the running head of the previous chapter suddenly reappears; so we need to use the command "\markboth{<left-mark>}{<right-mark>}" to customize the page header (the left-mark will normally be used in the header of even pages and right-mark in the header of odd pages; with one-sided printing, only the right-mark exists).
\markboth{Prefazione}{Prefazione}

%La prefazione offre al candidato uno spazio per includere informazioni che non si adattano facilmente agli altri capitoli della tesi; ad esempio, potrebbero essere esplorate le motivazioni personali che hanno guidato la scelta dell'argomento di ricerca.
%
%\medskip
%
%Questa sezione è anche ideale per evidenziare il contributo fornito dal relatore e da eventuali altri collaboratori; inoltre, se il lavoro di ricerca è stato svolto all'interno di un gruppo, la prefazione può essere utilizzata per chiarire lo specifico apporto fornito dello studente \cite{tuni2019guide}.
%
%\medskip
%
%In aggiunta, si possono inserire informazioni utili come:
%\begin{itemize}
%\item il software antiplagio impiegato per la verifica del documento;
%\item i link alle repository GitHub correlate al lavoro di tesi;
%\item la data dell'ultima revisione del testo.
%\end{itemize}

Durante l'anno accademico 2019-2020 ho collaborato con l'unità di ricerca \href{https://sysma.imtlucca.it/}{SySMA} della \href{https://www.imtlucca.it/it}{Scuola IMT Alti Studi Lucca} in qualità di beneficiario della borsa di ricerca \href{https://www.imtlucca.it/it/jobopportunity/verioss-smart-contract-development}{\textbf{VeriOSS smart contract development}} (finanziata con i fondi del progetto PAI 2018 "\textit{VeriOSS: a security-by-smart contract verification framework for Open Source Software}" - P0137). L'obiettivo della borsa era quello di progettare e sviluppare smart contract Solidity a supporto del protocollo di fair exchange di VeriOSS, una piattaforma per la bug bounty basata sulla blockchain. Il lavoro di tesi svolto prosegue e conclude quanto iniziato durante la suddetta collaborazione di ricerca.

\medskip

Tutto il materiale prodotto per questo lavoro di tesi è accessibile attraverso diverse repository pubbliche su GitHub; in particolare:

\begin{itemize}

\item i file \latex associati a questo documento di tesi si trovano in \myThesisRepo;

\item il codice implementato è disponibile in \myCodeRepo;

\item infine, i diagrammi di sequenza, di stato e di classe sono raccolti in \myDiagramsRepo.

\end{itemize}

Per individuare e correggere involontarie somiglianze o citazioni non adeguate, è stato utilizzato \myAntiplagio, il software antiplagio messo a disposizione dall'Università degli Studi di Firenze.