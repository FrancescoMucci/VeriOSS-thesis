\chapter*{Prefazione}
% The starred variants of all sectioning commands (\chapter*, \section*, ...) produce unnumbered headings which do not appear in the table of contents or in the page header. The absence of a running head often has an unwanted side effect: if, for example, a chapter set using \chapter* spans several pages, then the running head of the previous chapter suddenly reappears; so we need to use the command "\markboth{<left-mark>}{<right-mark>}" to customize the page header (the left-mark will normally be used in the header of even pages and right-mark in the header of odd pages; with one-sided printing, only the right-mark exists).
\markboth{Prefazione}{Prefazione}

La prefazione offre al candidato uno spazio per includere informazioni che non si adattano facilmente agli altri capitoli della tesi; ad esempio, potrebbero essere esplorate le motivazioni personali che hanno guidato la scelta dell'argomento di ricerca.

\medskip

Questa sezione è anche ideale per evidenziare il contributo fornito dal relatore e da eventuali altri collaboratori; inoltre, se il lavoro di ricerca è stato svolto all'interno di un gruppo, la prefazione può essere utilizzata per chiarire lo specifico apporto fornito dello studente \cite{tuni2019guide}.

\medskip

In aggiunta, si possono inserire informazioni utili come:
\begin{itemize}
\item il software antiplagio impiegato per la verifica del documento;
\item i link alle repository GitHub correlate al lavoro di tesi;
\item la data dell'ultima revisione del testo.
\end{itemize}

\section*{Prefazione d'esempio}

Ho sviluppato questo modello per la mia tesi di laurea magistrale partendo da \href{https://www.informaticamagistrale.unifi.it/vp-17-per-laurearsi.html}{quello fornito dal Corso di Laurea Magistrale in Informatica dell'Università degli Studi di Firenze}; quest'ultimo, a sua volta, è stato elaborato utilizzando come base il pacchetto \href{https://www.ctan.org/pkg/classicthesis}{\texttt{classicthesis}}, creato da \href{https://www.miede.de/}{André Miede}.

\medskip

La struttura (capitoli e sezioni) del modello è stata derivata a partire dalle indicazioni fornite da:
\begin{itemize}

\item Zobel in \textit{"Writing for Computer Science"} \cite{zobel2015writing};

\item Pfandzelter \etAl in \textit{"Writing a Computer Science Thesis"} \cite{pfandzelter2022thesis};

\item Shoaff in \textit{"How to Write a Master's Thesis in Computer Science"} \cite{shoaff2001thesis};

\item Männistö \etAl in \textit{"Scientific Writing - Guide of the Empirical Software Engineering Research Group of the University of Helsinki"} \cite{mannisto2022guide};

\item Aceto in \textit{"How to Write a Paper"} \cite{aceto2023paper};

\item la \textit{Tampere University} in \textit{"Guide to Writing a Thesis in Technical Fields"} \cite{tuni2019guide};

\item la \textit{Libera Università di Bolzano} in \textit{"Master in Computer Science - Guidelines for the Thesis"} \cite{unibz2022thesis};

\item la \textit{Harran University} in \textit{"Software Design Report"} \cite{harran2023design};

\item il laboratorio \textit{Computer Science 7 (Computer Networks and Communication Systems)} della \textit{Friedrich-Alexander Universität} sulla \href{https://www.cs7.tf.fau.eu/teaching/student-theses/writing-your-thesis/}{pagina dedicata alle linee guida per la tesi} \cite{fau2023thesis}.

\end{itemize}

Per individuare e correggere involontarie somiglianze o citazioni non adeguate, è stato utilizzato \myAntiplagio, il software antiplagio messo a disposizione dall'Università degli Studi di Firenze.

\medskip

I file \latex associati a questo documento sono disponibili nella seguente repository GitHub pubblica: \myThesisRepo.