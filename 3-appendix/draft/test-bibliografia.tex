\addChapterBookmark{Test della bibliografia}{bookmark:bibtest}
% Although we do not include an entry for the bibliography test in the table of contents itself, we create a bookmark for it at level 0.

\chapter*{Test della bibliografia}
\label{appendix:bib-test}

\acresetall
% Empty the memory of the "\ac" macro: each time you use an acronym for the first time, the full name and the short name in brackets will be printed; afterwards only the acronym will be printed.

%Le appendici dovrebbero includere materiale che, pur essendo necessario per comprendere il lavoro svolto, non rappresenta una parte centrale della tesi o non può essere inserito nel testo principale a causa delle sue dimensioni eccessive o del formato particolare \cite{tuni2019guide}.

\addSectionBookmark{Introduzione all'appendice}{bookmark:bibtest-intro}
\section*{Introduzione all'appendice}

%Poiché è possibile che il lettore consulti le appendici senza aver letto integralmente la tesi, è consigliabile includere in ognuna di queste una breve introduzione che ne descriva il contenuto e che la collochi nel contesto più ampio del lavoro svolto \cite{tuni2019guide}.

In questa appendice, riservata unicamente alla bozza della tesi, vengono presentati i riferimenti bibliografici consultati, organizzati in base all'argomento e alla tipologia di documento.

\addSectionBookmark{VeriOSS}{bookmark:bibtest-verioss}
\section*{VeriOSS}

\subsection*{Articoli scientifici di Costa et al.}
\begin{itemize}

\item VeriOSS: Using the Blockchain to Foster Bug Bounty Programs \cite{canidio2021verioss};

\item Verifying a Blockchain-Based Remote Debugging Protocol for Bug Bounty \cite{degano2021verioss}.

\end{itemize}

\addSectionBookmark{Piattaforme bug bounty}{bookmark:bibtest-bugbonty}
\section*{Piattaforme bug bounty}

\subsection*{Tesi di dottorato di Walshe e articoli scientifici di Walshe et al.}
\begin{itemize}

\item Supporting data-driven software development life-cycles with bug bounty programmes \cite{walshe2023bountythesis};

\item Current State of Bug Bounty Programmes and Platforms \cite{walshe2023bountythesis3};

\item An Empirical Study of Bug Bounty Programs \cite{walshe2020bountypaper}.

\end{itemize}

\subsection*{Articoli scientifici di Akgul et al.}
\begin{itemize}

\item Bug hunters' perspectives on the challenges and benefits of the bug bounty ecosystem \cite{akgul2023bughunters};

\item The Hackers' Viewpoint: Exploring Challenges and Benefits of Bug-Bounty Programs \cite{akgul2020bughunters}.

\end{itemize}

\subsection*{Altri articoli scientifici}
\begin{itemize}

\item Bug Bounty Programs for Cybersecurity: Practices, Issues, and Recommendations \cite{malladi2020bugbounty};

\item Web Science Challenges in Researching Bug Bounties \cite{fryer2017bugbounty}.

\end{itemize}

\addSectionBookmark{Piattaforme bug bounty basate sulle blockchain}{bookmark:bibtest-blockchainbounty}
\section*{Piattaforme bug bounty basate sulle blockchain}

\subsection*{Articoli scientifici di Hoffman et al. su Bountychain}
\begin{itemize}

\item Decentralized Security Bounty Management on Blockchain and IPFS \cite{hoffman2020bountychain};

\item Bountychain: Toward Decentralizing a Bug Bounty Program with Blockchain and IPFS \cite{hoffman2021bountychain}.

\end{itemize}

\subsection*{Articoli scientifici di Badash et al. su BBBB Framework}
\begin{itemize}

\item Blockchain-Based Bug Bounty Framework \cite{badash2021blockbounty}.

\end{itemize}

\subsection*{Articoli scientific di Lisi et al. su ARD}
\begin{itemize}

\item Automated Responsible Disclosure of Security Vulnerabilities \cite{lisi2022ard}.

\end{itemize}

\addSectionBookmark{Protocolli di fair exchange}{bookmark:bibtest-fairexchange}
\section*{Protocolli di fair exchange}

\subsection*{Articoli scientifici seminali}
\begin{itemize}

\item Optimistic Protocols for Multi-Party Fair Exchange \cite{asokan1996fairexchange};

\item Fair exchange with a semi-trusted third party \cite{franklin1997fairexchange};

\item Optimistic fair exchange of digital signatures \cite{asokan1998fairexchange};

\item Secure group barter: Multi-party fair exchange with semi-trusted neutral parties \cite{franklin1998fairexchange}.

\end{itemize}

\subsection*{Articoli scientifici panoramici}
\begin{itemize}

\item A review of fair exchange protocols \cite{alotaibi2012fairreview};

\item A survey on optimistic fair exchange protocol and its variants \cite{loh2017fairreview};

\item Fair Exchange Protocol in Electronic Transactions Revisited \cite{duangphasuk2020fairreview}.

\end{itemize}

\addSectionBookmark{Protocolli di fair exchange basati sulla blockchain}{bookmark:bibtest-fairchain}
\section*{Protocolli di fair exchange basati sulla blockchain}

\subsection*{Articoli scientifici su FairSwap}
\begin{itemize}

\item FairSwap: How To Fairly Exchange Digital Goods \cite{dziembowski2018fairswap};

\item Privacy-preserving FairSwap: Fairness and privacy interplay \cite{avizheh2022fairswap}.

\end{itemize}

\subsection*{Articoli scientifici su OptiSwap}
\begin{itemize}

\item OptiSwap: Fast Optimistic Fair Exchange \cite{eckey2020optiswap};

\item Privacy-enhanced OptiSwap \cite{avizheh2021optiswap}.

\end{itemize}

\subsection*{Articoli scientifici su cost fairness}
\begin{itemize}

\item Cost Fairness for Blockchain-Based Two-Party Exchange Protocols \cite{lhor2020costfairness};

\item Formalizing Cost Fairness for Two-Party Exchange Protocols using Game Theory and Applications to Blockchain \cite{lhor2022costfairness};

\item Formalizing Cost Fairness for Two-Party Exchange Protocols using Game Theory and Applications to Blockchain (Extended Version) \cite{lhor2022costfairnessext}.

\end{itemize}

\subsection*{Articoli scientifici su protocolli che usano zero-knowledge proof}
\begin{itemize}

\item FileBounty: Fair Data Exchange \cite{janin2020fairdata};

\item Contingent payments from two-party signing and verification for abelian groups \cite{bursuc2022fairabelian}.

\end{itemize}

\subsection*{Altri articoli scientifici}
\begin{itemize}

\item FairTrade: Efficient Atomic Exchange-based Fair Exchange Protocol for Digital Data Trading \cite{chenli2021fairtrade}.

\end{itemize}

\addSectionBookmark{Proof of Knowledge}{bookmark:bibtest-pok}
\section*{Proof of Knowledge}

\subsection*{Monografie}
\begin{itemize}

\item Proofs, Arguments, and Zero-Knowledge \cite{thaler2022pokbook}.

\end{itemize}

\subsection*{Capitoli di libri}
\begin{itemize}

\item Sigma Protocols and Efficient Zero-Knowledge \cite{hazay2010sigmazero};

\item Identification and signatures from Sigma protocols \cite{boneh2023sigma};

\item Proving properties in zero-knowledge \cite{boneh2023zero};

\item A Survey on Zero-Knowledge Proofs \cite{li2014zero}.

\end{itemize}

\subsection*{Articoli scientifici seminali}
\begin{itemize}

\item The Knowledge Complexity of Interactive Proof-Systems \cite{goldwasser1985pok}.

\end{itemize}

\subsection*{Articoli scientifici}
\begin{itemize}

\item Do You Need a Zero Knowledge Proof? \cite{ernstberger2024zeroneed};

\item A survey on zero knowledge range proofs and applications \cite{morais2019zero}.

\end{itemize}

\addSectionBookmark{Proof of Knowledge per la blockchain}{bookmark:bibtest-pokchain}
\section*{Proof of Knowledge per la blockchain}

\subsection*{Articoli scientifici panoramici}
\begin{itemize}

\item Overview of Zero-Knowledge Proof and Its Applications in Blockchain \cite{zhou2022zerochain};

\item Non-Interactive Zero-Knowledge for Blockchain: A Survey \cite{partala2020zerochain};

\item A Survey on Zero-Knowledge Proof in Blockchain \cite{sun2021zerochain}.

\end{itemize}

\addSectionBookmark{Blockchain, Ethereum e Solidity}{bookmark:bibtest-ethereum}
\section*{Fondamenti di Blockchain, Ethereum e Solidity}

\subsection*{Libri generici sulla blokchain}
\begin{itemize}

\item Handbook on Blockchain \cite{tran2022blockchain};

\item Blockchain Essentials - Core Concepts and Implementations \cite{mangrulkar2024blockchain}.

\end{itemize}

\subsection*{Documentazione di Ethereum e Solidity}
\begin{itemize}

\item Ethereum Development Documentation \cite{ethereum2024doc};

\item Solidity Documentation - Release 0.8.18 \cite{solidity0.8.18doc}.

\end{itemize}

\subsection*{Libri specifici per Ethereum e sviluppo di smart contracts Solidity}
\begin{itemize}

\item Mastering Ethereum: Building Smart Contracts and DApps \cite{antonopoulos2018ethereum};

\item Ethereum Smart Contract Development in Solidity \cite{zheng2020solidity};

\item Blockchain and Ethereum Smart Contract Solution Development - Dapp Programming with Solidity \cite{zhang2022solidity};

\item Solidity Programming Essentials: A guide to building smart contracts and tokens using the widely used Solidity language \cite{modi2022solidity}.

\end{itemize}

\addSectionBookmark{Architettura e sviluppo di applicazioni blockchain-based}{bookmark:blockchain-based-dev}
\section*{Architettura e sviluppo di applicazioni blockchain-based}

\subsection*{Libro e articoli scientifici di Xu et al.}
\begin{itemize}

\item Architecture for Blockchain Applications \cite{xu2019book};

\item A Pattern Collection for Blockchain-based Applications \cite{xu2018patterns};

\item Applying Design Patterns in Smart Contracts \cite{xu2018design};

\item A Taxonomy of Blockchain-Based Systems for Architecture Design \cite{xu2017architecture}.

\end{itemize}

\subsection*{Tesi di dottorato di Wöhrer e articoli scientifici di Wöhrer et al.}
\begin{itemize}

\item Engineering Blockchain-Based Applications in the Context of the Ethereum Ecosystem \cite{wohrer2022thesis};

\item Design Patterns for Smart Contracts in the Ethereum Ecosystem \cite{wohrer2018designpatterns};

\item Smart contracts: security patterns in the ethereum ecosystem and solidity\cite{wohrer2018securitypatterns};

\item Foundational Oracle Patterns: Connecting Blockchain to the Off-Chain World \cite{wohrer2020oracle};

\item Architectural Design Decisions for Blockchain-Based Applications \cite{wohrer2021decisions};

\item Architecture Design of Blockchain-Based Applications \cite{wohrer2021architecture}.

\end{itemize}

\subsection*{Articoli scientifici di Marchesi et al.}
\begin{itemize}

\item Design Patterns for Gas Optimization in Ethereum \cite{marchesi2020gas};

\item ABCDE--agile block chain DApp engineering \cite{marchesi2020agile};

\item An Agile Software Engineering Method to Design Blockchain Applications \cite{marchesi2018agile}.

\end{itemize}

\subsection*{Altri articoli scientifici}
\begin{itemize}

\item Towards saving money in using smart contracts \cite{chen2018gas};

\item A Study of Blockchain Oracles \cite{beniiche2020oracle};

\item Do you Need a Blockchain? \cite{wurst2018blockchain};

\item Challenges and Common Solutions in Smart Contract Development \cite{kannengiesser2022challanges}.

\end{itemize}

\addSectionBookmark{Verifica formale di smart contract}{bookmark:smartcontract-formalver}
\section*{Verifica formale di smart contract}

\subsection*{Documentazione e articoli scientifici panoramici}

\begin{itemize}

\item Formal Verification of Smart Contracts \cite{ethereum2024scfv};

\item A Survey of Smart Contract Formal Specification and Verification \cite{tolmach2021scfsvsurvey};

\item Formal Methods for the Verification of Smart Contracts: A Review \cite{krichen2022scfmvreview};

\end{itemize}

\subsection*{Documentazione e articoli scientifici su SMTChecker di Solidity}

\begin{itemize}

\item Solidity Documentation - Release 0.8.18, Sezione 3.30 - SMTChecker and Formal Verification \cite{solidity0.8.18smtcheckerdoc}

\item A Solicitous Approach to Smart Contract Verification \cite{otoni2023verification}.

\end{itemize}

\clearpage
\addSectionBookmark{Debugging}{bookmark:debugging}
\section*{Debugging}

\subsection*{Revisioni sistematiche}
\begin{itemize}

\item Debugging: a review of the literature from an educational perspective \cite{mccauley2008debugreview};

\item A Systematic Review on Program Debugging Techniques \cite{ghosh2019debugreview}.

\end{itemize}

\subsection*{Remote debugging}
\begin{itemize}

\item Mercury: Properties and Design of a Remote Debugging Solution using Reflection \cite{papoulias2015remotedebug};

\item Remote Debugging for Containerized Applications in Edge Computing Environments \cite{ozcan2019remotedebug}.

\end{itemize}

\subsection*{Reverse debugging}
\begin{itemize}

\item A review of reverse debugging \cite{engblom2012reversedebug};

\item Implementation of Live Reverse Debugging in LLDB \cite{savidis2021reversedebug}.

\end{itemize}

\addSectionBookmark{Weakest precondition calculus}{bookmark:wpc}
\section*{Weakest precondition calculus}

\subsection*{Articoli scientifici seminali}
\begin{itemize}

\item Guarded commands, nondeterminacy and formal derivation of programs \cite{dijkstra1975wpc}.

\end{itemize}

\subsection*{Libri di testo}
\begin{itemize}

\item A Discipline of Programming \cite{dijkstra1976wpcbook};

\item The Science of Programming \cite{gries1987wpcbook};

\item Predicate Calculus and Program Semantics \cite{dijkstra1990wpcbook}.

\end{itemize}

\subsection*{Altri articoli scientifici}
\begin{itemize}

\item The Weakest Precondition Calculus: Recursion and Duality \cite{bonsangue1994wpc}.

\end{itemize}

\addSectionBookmark{Symbolic execution}{bookmark:symbexe}
\section*{Symbolic execution}

\subsection*{Revisioni sistematiche}
\begin{itemize}

\item A Survey of Symbolic Execution Techniques \cite{baldoni2018symbexereview};

\item Advances in Symbolic Execution \cite{yang2019symbexereview};

\item Symbolic Execution and Recent Applications to Worst-Case Execution, Load Testing, and Security Analysis \cite{pasareanu2019symbexereview}.

\end{itemize}

\subsubsection*{Revisioni di tools}
\begin{itemize}

\item Benchmarking the Capability of Symbolic Execution Tools with Logic Bombs \cite{xu2020symbexetools};

\item Systematic comparison of symbolic execution systems: intermediate representation and its generation \cite{poeplau2019symbexetools}.

\end{itemize}

\subsection*{Altri articoli scientifici}
\begin{itemize}

\item Symbolic execution formally explained \cite{boer2021symbexe}.

\end{itemize}

\addSectionBookmark{Backward symbolic execution}{bookmark:backsymbexe}
\section*{Backward symbolic execution}

\subsection*{Backward symbolic execution via weakest precondition calculus}
\begin{itemize}

\item Snugglebug: a powerful approach to weakest preconditions \cite{chandra2009bsewpc};

\item Handling Heap Data Structures in Backward Symbolic Execution \cite{husak2020bsewpc};

\item Higher-order demand-driven symbolic evaluation \cite{zachary2020bsewpc};

\item Backward Symbolic Execution with Loop Folding \cite{chalupa2021bsewpc};

\item Generation of the weakest preconditions of programs with dynamic memory in symbolic execution \cite{misonizhnik2022bsewpc}.

\end{itemize}