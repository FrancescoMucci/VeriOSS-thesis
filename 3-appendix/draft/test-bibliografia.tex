\addChapterBookmark{Test della bibliografia}{bookmark:bibtest}
% Although we do not include an entry for the bibliography test in the table of contents itself, we create a bookmark for it at level 0.

\chapter*{Test della bibliografia}
\label{appendix:bib-test}

\acresetall
% Empty the memory of the "\ac" macro: each time you use an acronym for the first time, the full name and the short name in brackets will be printed; afterwards only the acronym will be printed.

%Le appendici dovrebbero includere materiale che, pur essendo necessario per comprendere il lavoro svolto, non rappresenta una parte centrale della tesi o non può essere inserito nel testo principale a causa delle sue dimensioni eccessive o del formato particolare \cite{tuni2019guide}.

\addSectionBookmark{Introduzione all'appendice}{bookmark:bibtest-intro}
\section*{Introduzione all'appendice}

%Poiché è possibile che il lettore consulti le appendici senza aver letto integralmente la tesi, è consigliabile includere in ognuna di queste una breve introduzione che ne descriva il contenuto e che la collochi nel contesto più ampio del lavoro svolto \cite{tuni2019guide}.

In questa appendice, riservata unicamente alla bozza della tesi, vengono presentati i riferimenti bibliografici consultati, organizzati in base all'argomento e alla tipologia di documento.

\addSectionBookmark{VeriOSS}{bookmark:bibtest-verioss}
\section*{VeriOSS}

\subsection*{Articoli scientifici di Costa et al.}

\begin{itemize}

\item VeriOSS: Using the Blockchain to Foster Bug Bounty Programs \cite{canidio2021verioss};

\item Verifying a Blockchain-Based Remote Debugging Protocol for Bug Bounty \cite{degano2021verioss}.

\end{itemize}

\addSectionBookmark{Piattaforme bug bounty}{bookmark:bibtest-bugbonty}
\section*{Piattaforme bug bounty}

\subsection*{Tesi di dottorato di Walshe e articoli scientifici di Walshe et al.}

\begin{itemize}

\item Supporting data-driven software development life-cycles with bug bounty programmes \cite{walshe2023bountythesis};

\item Current State of Bug Bounty Programmes and Platforms \cite{walshe2023bountythesis3};

\item An Empirical Study of Bug Bounty Programs \cite{walshe2020bountypaper}.

\end{itemize}

\subsection*{Articoli scientifici di Akgul et al.}

\begin{itemize}

\item Bug hunters' perspectives on the challenges and benefits of the bug bounty ecosystem \cite{akgul2023bughunters};

\item The Hackers' Viewpoint: Exploring Challenges and Benefits of Bug-Bounty Programs \cite{akgul2020bughunters}.

\end{itemize}

\subsection*{Altri articoli scientifici}

\begin{itemize}

\item Bug Bounty Programs for Cybersecurity: Practices, Issues, and Recommendations \cite{malladi2020bugbounty};

\item Web Science Challenges in Researching Bug Bounties \cite{fryer2017bugbounty}.

\end{itemize}

\addSectionBookmark{Piattaforme bug bounty basate sulle blockchain}{bookmark:bibtest-blockchainbounty}
\section*{Piattaforme bug bounty basate sulle blockchain}

\subsection*{Articoli scientifici di Hoffman et al. su Bountychain}

\begin{itemize}

\item Decentralized Security Bounty Management on Blockchain and IPFS \cite{hoffman2020bountychain};

\item Bountychain: Toward Decentralizing a Bug Bounty Program with Blockchain and IPFS \cite{hoffman2021bountychain}.

\end{itemize}

\subsection*{Articoli scientifici di Badash et al. su BBBB Framework}

\begin{itemize}
\item Blockchain-Based Bug Bounty Framework \cite{badash2021blockbounty}.
\end{itemize}

\subsection*{Articoli scientific di Lisi et al. su ARD}

\begin{itemize}
\item Automated Responsible Disclosure of Security Vulnerabilities \cite{lisi2022ard}.
\end{itemize}

\addSectionBookmark{Blockchain, Ethereum e Solidity}{bookmark:bibtest-ethereum}
\section*{Blockchain, Ethereum e Solidity}

\subsection*{Libri generici sulla blokchain}

\begin{itemize}

\item Handbook on Blockchain \cite{tran2022blockchain};

\item Blockchain Essentials - Core Concepts and Implementations \cite{mangrulkar2024blockchain};

\end{itemize}

\subsection*{Libri sull'archittettura di applicazioni basate sulla blockchain}

\begin{itemize}

\item Architecture for Blockchain Applications \cite{xu2019blockarch}.

\end{itemize}

\subsection*{Documentazione di Ethereum e Solidity}

\begin{itemize}

\item Ethereum Development Documentation \cite{ethereum2024doc};

\item Solidity Documentation - Release 0.8.18 \cite{solidity0.8.18doc};

\end{itemize}

\subsection*{Libri specifici per Ethereum e sviluppo di smart contracts Solidity}

\begin{itemize}

\item Mastering Ethereum: Building Smart Contracts and DApps \cite{antonopoulos2018ethereum};

\item Ethereum Smart Contract Development in Solidity \cite{zheng2020solidity};

\item Blockchain and Ethereum Smart Contract Solution Development - Dapp Programming with Solidity \cite{zhang2022solidity}

\item Solidity Programming Essentials: A guide to building smart contracts and tokens using the widely used Solidity language \cite{modi2022solidity}

\end{itemize}

\addSectionBookmark{Verifica formale di smart contract}{bookmark:smartcontract-formalver}
\section*{Verifica formale di smart contract}

\subsection*{Documentazione e articoli scientifici che forniscono una panoramica}

\begin{itemize}

\item Formal Verification of Smart Contracts \cite{ethereum2024scfv};

\item A Survey of Smart Contract Formal Specification and Verification \cite{tolmach2021scfsvsurvey};

\item Formal Methods for the Verification of Smart Contracts: A Review \cite{krichen2022scfmvreview};

\end{itemize}

\subsection*{Documentazione e articoli scientifici su SMTChecker di Solidity}

\begin{itemize}

\item Solidity Documentation - Release 0.8.18, Sezione 3.30 - SMTChecker and Formal Verification \cite{solidity0.8.18smtcheckerdoc}

\item A Solicitous Approach to Smart Contract Verification \cite{otoni2023verification}.

\end{itemize}

\addSectionBookmark{Protocolli di fair exchange}{bookmark:bibtest-fairexchange}
\section*{Protocolli di fair exchange}
